\documentclass[12pt,a4paper,ngerman]{article}
\usepackage{amsmath}
\usepackage{amsthm}
\usepackage[utf8]{inputenc}
%Layout
\setlength{\topmargin}{1pt} \setlength{\headheight}{0.3cm}
\setlength{\textwidth}{16.5cm} \setlength{\textheight}{23cm}
\setlength{\oddsidemargin}{0cm} \setlength{\evensidemargin}{0cm}
\setlength{\headsep}{2pt} \setlength{\parindent}{0pt}
\setlength{\marginparwidth}{80pt} \pagestyle{plain}

\title{Physik TE \\
Ausarbeitung und Zusammenfassung}
\author{Martin Winter}
\date{08.09.2014 - 14.09.2014}

\begin{document}
  \maketitle
\begin{abstract}
Die folgende Zusammenfassung soll als Hilfsmittel dienen um die Physik TE Prüfung erfolgreich abzuschließen. 
Es werden dabei das Skriptum sowie die Ausarbeitungen von Bernhard Geiger, Matthias Straka sowie der Fragenkatalog von der PBS verwendet, hiermit möchte ich mich bei den Urhebern dieser Dokumente herzlichst bedanken. 
\end{abstract}
\pagebreak
  
\tableofcontents
  
\pagebreak
  
\renewcommand{\arraystretch}{1.5}
%-------------------------------------------------------------------------------
%
% Beginn der Zusammenfassung
%
%-------------------------------------------------------------------------------

\section{Mechanik}
\subsection{Basiseinheiten}
Dieses Thema kommt gerne als Prüfungsfrage, hier sind die wichtigsten gelistet:
\\
\\
\textbf{Länge}: Ein Meter ist die Länge der Strecke, die Licht im Vakuum während der Dauer $\frac{1}{c} = \frac{1}{299792458} $ s durchläuft, c steht dabei für die \textit{Lichtgeschwindigkeit} und beträgt etwas weniger als $300000000$ m/s.
\vspace{0.5cm}\\
\textbf{Zeit}: Eine Sekunde ist das $9192631770$-fache der Periodendauer der elektromagnetischen Strahlung des Übergangs zwischen den Hyperfeinstruktur-Niveaus des von äußeren Feldern ungestörten Cäsium-Isotops $^{133}Cs$.
\\
Länge und Zeit können mit einer Unsicherheit von $10^{-14}$ bestimmt werden, Masse nur mit einer Unsicherheit von $10^{-9}$. 
\vspace{0.5cm}\\
\textbf{Masse}: $1$ kg ist die Masse eines Platin-Iridium-Zylinders, der als Massennormal in Paris aufbewahrt wird.

\subsection{Kinematik des Massenpunktes}
Diese beschäftigt sich mit der Beschreibung von Bahnkurven und dem zeitlichen Ablauf einer Bewegung. \\
\subsubsection*{Bewegung auf geradliniger Bahn}
Geschwindigkeit ist definiert als
\begin{equation}
v = \frac{\text{zurückgelegte Wegstrecke } s}{\text{dabei verstrichene  Zeit }  t} = 
\frac{\Delta s}{\Delta t}, [m/s]
\end{equation}
Die Beschleunigung ist definiert als
\begin{equation}
a = \frac{\text{Änderung der Geschwindigkeit }v}{\text{dabei verstrichene Zeitdauer }t} = \frac{\Delta v}{\Delta t}, [m/s^2]
\end{equation}

Der einfachste Fall einer ungleichförmigen Bewegung ist somit die gleichförmig beschleunigte Bewegung
\begin{equation}
v = a \cdot t , \quad s = \frac{a\cdot t^2}{2} = \frac{v\cdot t}{2} \text{ sowie } v = \sqrt{2\cdot a \cdot s}
\end{equation}
\subsubsection*{Bewegung auf krummliniger Bahn}
Definition des Winkels im Bogenmaß
\begin{equation}
\varphi = \frac{\text{Kreisbogen }b}{\text{Kreisradius }r} [m/m], \text{Radiant}
\end{equation}
Definition der Winkelgeschwindigkeit
\begin{equation}
\omega= \frac{\text{Änderung des Winkels }\varphi}{\text{dabei verstrichene Zeitdauer }t} = \frac{\Delta \varphi}{\Delta t}
\end{equation}
Bei einer gleichförmigen Bewegung kann man die Anzahl der Umläufe pro Sekunde als Frequenz definieren
\begin{equation}
\nu = \frac{1}{\tau}, [s^{-1}], \text{Hz(Hertz)}
\end{equation}
Dabei wird die Zeit für einen vollständigen Umlauf als \textit{Periodendauer} $\tau$ bezeichnet, bei einer gleichförmigen Bewegung gilt somit
\begin{equation}
\omega = 2 \pi \nu
\end{equation}

\subsection{Dynamik}
\subsubsection*{Kräfte}
Eine Kraft wird definiert als
\begin{equation}
\vec{F} = m \cdot \vec{a}, [kg\frac{m}{s^2}] \text{ Newton}
\end{equation}
Newton erkannte, dass zwischen Körpern \textit{Wechselwirkungen} herrschen. Als \textit{Kraft} bezeichnet man die Änderung des Bewegungszustand eines Körpers. Ein Körper ohne Wechselwirkungen (auf den keine Kräfte wirken oder wenn die Vektorsumme aller Kräfte Null ist) wird \textit{frei} genannt und ändert seinen Bewegungszustand nicht.
Kräfte können zum Beispiel gemessen werden durch \textit{Verformung} einer Federwage, es gilt dabei
\begin{equation}
F_x = D(x-x_0), \text{ wobei D...Federkonstante und } x-x_0 \text{...Auslenkung entspricht}
\end{equation}
Ein \textit{Wiegen} von Körper mittels einer Balkenwage wird als \textit{Massevergleich} und nicht als \textit{Kraftbestimmung}
bezeichnet.
Die auf einen Körper wirkenden Kräfte sind meist vom Ort abhängig, somit kann jedem Punkt im Raum ein Kraftvektor zugeordnet werden, man spricht dann von einem \textit{Kraftfeld}. \\
Das \textbf{Gravitationsfeld der Erde} ist gegeben als
\begin{equation}
\vec{F} = -G \frac{m \cdot M}{r^2}\vec{e_r} \text{ wobei M...Erdmasse, m...Körpermasse und G...Gravitationskonstante}
\end{equation}
und das \textbf{Kraftfeld einer elektrischen Ladung} ist gegeben als
\begin{equation}
\vec{F} = \frac{1}{4 \pi \epsilon_0}\frac{q \cdot Q}{r^2}\vec{e_r}
\end{equation}
dabei erhält man ein Coulombfeld, dass dem Schwerefeld einer kugelförmigen Masse entspricht. \vspace{0.5cm}\\
\textbf{Homogenes Kraftfeld eines Plattenkondensators} Die Feldlinien sind hierbei parallel zueinander und haben die gleiche Richtung, ein solches Feld wird \textbf{homogen} genannt. 

\subsubsection*{Grundgleichungen der Mechanik}

\textbf{1. NEWTON'sches Axiom}:

\begin{verse}
Jeder Körper verharrt im \textbf{Zustand der Ruhe} oder der gleichförmigen, geradlinigen Bewegung, solange \textbf{keine Kraft} auf ihn einwirkt. 
\end{verse}

\vspace{0.5cm}

\textbf{2. NEWTON'sches Axiom}:

\begin{verse}
Eine \textbf{zeitliche Änderung} der Bewegungsgröße ist der bewegenden Kraft, durch die sie verursacht wird, \textbf{proportional} und verläuft \textbf{in Richtung der Kraft.}
\end{verse}

\vspace{0.5cm}

\textbf{3. NEWTON'sches Axiom}:

\begin{verse}
Bei zwei Körpern, die \textbf{nur miteinander}, aber \textbf{nicht mit anderen Körpern} wechselwirken, ist die Kraft $\vec{F_1}$ auf den einen Körper entgegengesetzt gleich der Kraft $\vec{F_2}$ auf den anderen Körper. \\
\begin{center}
\textbf{Actio = Reactio: $F_1 = F_2$}
\end{center}
\end{verse}

\vspace{0.5cm}

Als Maß für den Bewegungszustand führt man eine Bewegungsgröße ein, den \textbf{Impuls}
\begin{equation}
p = m \cdot v , \quad p = [kgms^{-1}]
\end{equation}
Die Impulsänderung wirkt als Kraft und ist definiert als
\begin{equation}
F = \frac{dp}{dt} \text{und wegen } p = m \cdot v \text{ gilt } F = m \cdot \frac{dv}{dt} + \frac{dm}{dt}\cdot v
\end{equation}
Ist die \textbf{Masse zeitlich konstant}, so erhält man wiederum 
\begin{equation}
F = m \cdot a
\end{equation}
Ein \textbf{Kraftstoß} ist eine zeitabhängige Kraft F und ruft eine \textbf{Impulsänderung} hervor. \\
Da auf ein abgeschlossenes System keine äußeren Kräfte wirken, kann man daraus schließen, dass der Gesamtimpuls des Systems erhalten bleibt, es gilt also
\begin{equation}
p_1 + p_2 = const
\end{equation}
\begin{verse}
In einem abgeschlossenen System bleibt der \textbf{Gesamtimpuls konstant}. 
\end{verse}
\textbf{Träge und schwere Masse}: \textit{Trägheit} ist die Eigenschaft einer Masse, in ihrem Bewegungszustand zu verharren, wenn keine Kraft auf sie wirkt. Die Kraft zur Änderung ist proportional zur Masse. Schwere Masse ist das Gewicht hervorgerufen durch die Gravitationskraft. \\
Schwere und träge Masse ist nicht unterscheidbar. 

\pagebreak

\subsubsection*{Der Energiesatz der Mechanik}
Legt ein Massepunkt in einem Kraftfeld F(r) das Wegelement $\Delta r$ zurück, so nennen wir das Skalarprodukt
\begin{equation}
\Delta W = \vec{F}(r) \cdot \Delta \vec{r}, \quad W = [Nm] = \text{Joule}
\end{equation}
die \textbf{mechanische Arbeit}, die von der Kraft F am Massepunkt entlang des Weges geleistet wird. Die Arbeit ist eine \textbf{skalare Größe}, die sich aus dem Skalarprodukt zweier Vektoren berechnen lässt
\begin{equation}
W = \int_{P_1}^{P_2}{\vec{F}(r) \cdot d\vec{r}}
\end{equation}
Die Arbeit pro Zeiteinheit nennt man die \textbf{Leistung}
\begin{equation}
P = \frac{dW}{dt}, P = [J/s] = \text{W(Watt)}
\end{equation}

\vspace{0.5cm}
\textbf{Wegunabhängige Arbeit und konservative Kraftfelder}\\
\begin{verse}
In konservativen Kraftfeldern ist die Arbeit bei der Bewegung eines Massenpunktes auf einem geschlossenen Weg Null und hängt sonst nur von Anfangs- und Endpunkt, nicht aber dem gewählten Weg ab. 
\end{verse}

\begin{equation}
W = \int_{P_1}{P_2}{\vec{F}(r) \cdot d\vec{r}} = E_{pot}(P_1) - E_{pot}(P_2)
\end{equation}
Die Kraft, die notwendig ist, um einen Massenpunkt von $P_1$ nach $P_2$ zu verschieben entspricht also der \textbf{Differenz der potentiellen Energie}. Die potentielle Energie in Abhängigkeit von den Ortskoordinaten wird als \textbf{Potential} bezeichnet.
\begin{equation}
E_{pot}(\vec{r})= m \cdot V_{pot}(\vec{r}) = m \cdot g \cdot h
\end{equation}
Der Nullpunkt der potentiellen Energie ist jeweils Definitionssache, dieser wird also meistens am Boden oder im Unendlichen angenommen. \\
Die Gesamtenergie bleibt im Fall konservativer Kräfte erhalten, es gilt 
\begin{equation}
E = E_{pot} + E_{kin} = const
\end{equation}
Der \textbf{Zusammenhang zwischen Kraft und Potential} bei konservativen Kraftfeldern gibt an, dass eine ortsabhängige Kraft vorhanden sein muss
\begin{equation}
\vec{F}  = -m \cdot grad \ V_{pot}
\end{equation}


\pagebreak


\subsubsection*{Drehimpuls und Drehmoment}
Der Drehimpuls ist definiert als
\begin{equation}
\vec{L} = (\vec{r} \times \vec{p})
\end{equation}
Bei einer Bewegung in einer Ebene zeigt der Drehimpuls immer in die Normalrichtung senkrecht zur Ebene. \\

Das Drehmoment ist definiert als
\begin{equation}
\vec{D} = (\vec{r} \times \vec{F}) = \frac{d\vec{L}}{dt}
\end{equation}
es gilt also auch, dass die \textbf{zeitliche Änderung des Drehimpulses gleich dem wirkenden Drehmoment} ist. Es gelten auch Erhaltungssätze für den Drehimpuls. \\
Der \textbf{Eigendrehimpuls} bezieht sich auf die Rotation um eine Achse durch den Körperschwerpunkt, er ist gegeben durch das Produkt von \textbf{Trägheitsmoment I} und \textbf{Winkelgeschwindigkeit $\omega$}. 
\begin{equation}
\vec{L}_{eigen} = I \cdot \omega
\end{equation}
Der \textbf{Bahndrehimpuls} bezieht sich auf den Nullpunkt des Koordinatensystems und ist identisch mit dem Drehimpuls. 

\subsubsection*{KEPLER'sche Gesetze}

\begin{enumerate}
\item Die Bahnen der Planeten sind Ellipsen, in deren Brennpunkt die Sonne steht \\
\item Der Fahrstrahl (Radiusvektor) überstreicht in gleichen Zeiten gleiche Flächen. \\
\item Die Quadrate der Umlaufzeiten zweier Planeten verhalten sich wie die Kuben der großen Ellipsenhalbachsen. 
\end{enumerate}

\subsubsection*{Stoßvorgänge}
Als \textbf{Stoß} bezeichnet man eine kurzzeitige Kraftwirkung zwischen zwei relativ zueinander bewegten Körpern. \\
\textbf{Elastische Stöße}: Hierbei bleibt die gesamte kinetische energie der stoßenden Körper erhalten. 
Es gilt also neben dem Impulssatz auch der Energiesatz der Mechanik 
\begin{equation}
\frac{m_1\cdot v_1^2}{2} + \frac{m_2 \cdot v_2^2}{2} = \frac{m_1 \cdot v_1^{\prime2}}{2} + \frac{m_2 \cdot v_2^{\prime2}}{2}
\end{equation}
\textbf{Unelastische Stöße}: Hierbei gilt zwar der Impulssatz, aber nicht der Energiesatz der Mechanik, ein bestimmter Energieanteil wird also in Wärme oder eine andere Form umgewandelt. Es kommt also ein Summand Q dazu, der Verluste widerspiegelt. 

\begin{equation}
\frac{m_1\cdot v_1^2}{2} + \frac{m_2 \cdot v_2^2}{2} = \frac{(m_1 + m_2)  v^{\prime2}}{2} + Q
\end{equation}

\pagebreak

\section{Elektrizitätslehre}
\subsection{Elektrostatik}
Alle Ladungen sind Vielfache der \textit{Elementarladung}, die Einheit ist Coulomb: 1 C = 1 As. Die Kraft zwischen zwei Ladungen ist definiert als
\begin{equation}
\vec{F} = \frac{1}{4 \pi \epsilon_0}\frac{Q_1 \cdot Q_2}{r^2} \vec{r}_e, \quad \epsilon_0 = 8.854\cdot 10^{-12}\frac{As}{Vm}
\end{equation}
$\epsilon_0$ wird dabei als \textbf{Dielektrizitätskonstante} bezeichnet. \\
Die Kraft auf eine Einheitsladung wird bezeichnet als \textbf{elektrische Feldstärke}
\begin{equation}
\vec{E} = \frac{1}{4 \pi \epsilon_0} \frac{Q}{r^2}\vec{r}_e \left[\frac{V}{m}\right]
\end{equation}
\begin{center}
\textit{
Eine Punktladung Q erzeugt den Fluss $\Phi_{el} = \int_A{\vec{E}\cdot d\vec{A}} =  \frac{Q}{\epsilon_0}$ durch eine die Ladung umgebende Kugeloberfläche. Ist in einer geschlossenen Fläche keine Ladung enthalten, so ist auch der Fluss Null. } 
\end{center}
Die im Raum verteilten Ladungen sind Quellen und Senken des elektrischen Feldes. Da es sich beim Kraftfeld um ein konservatives Kraftfeld handelt, kann man eine Funktion aufstellen, die man als \textbf{elektrostatisches Potential} bezeichnet. 
\begin{equation}
V(P) = \int_{P_1}^{\infty}{\vec{E}\cdot d\vec{s}}
\end{equation}
Die Potentialdifferenz zwischen zwei Punkten im elektrischen Feld nennt man die Spannung U
\begin{equation}
U = V(P_1) - V(P_2) = \int_{P_1}^{P_2}{\vec{E}\cdot d\vec{s}}, \quad U = \text{V(Volt)}
\end{equation}
Die Feldstärke lässt sich auch durch \textbf{Gradientenbildung} aus dem Potential berechnen
\begin{equation}
\vec{E} = -grad \ V
\end{equation}

\subsubsection*{Kräfte auf einen Dipol}
In einem \textbf{homogenen Feld}, in dem sich ein Dipol mit zwei unterschiedlichen Ladungen befindet, sind die Kräfte der beiden Enden entgegengesetzt gerichtet $\Rightarrow$ Drehmoment, dass das Dipol parallel nach dem Feld ausrichtet. 

\subsubsection*{Leiter im elektrischen Feld}
Zwei entgegengesetzte Leiterplatten werden als \textbf{Kondensator} bezeichnet, es gilt dabei 
\begin{equation}
Q = C \cdot U, \quad C = \epsilon_0\frac{A}{d} \ [F]\text{(Farad)}
\end{equation}
Für die Energie im elektrischen Feld erhält man die Gleichung
\begin{equation}
W = \int{V \cdot dQ} = \frac{1}{2}C \cdot U^2
\end{equation}

\pagebreak

\subsubsection*{Millikan Versuch zur Ladungsbestimmung}
Es wurden dabei Öltröpfchen zwischen zwei horizontal angebrachte Kondensatorplatten gebracht, einige Tröpfchen wurden durch Reibung elektrisch geladen und trugen dadurch die gequantelte Ladung $q = n \cdot e$. Im feldfreien Kondensator sinken die Tröpfchen, erst durch Anlegen einer Spannung können diese in der Schwebe gehalten werden und geben dadurch Aufschluss auf deren Ladung, da sich Schwerkraft und elektrische Kraft aufheben müssen. 
Nach Messung der Größe der Tröpfchen lässt sich die \textbf{Elementarladung e} berechnen ($e = 1.602\cdot 10^-19 C$).

\subsubsection*{Ablenkung von $e^-$ in elektrischen Feldern}
Ein Teilchen mit Ladung \textit{q} fliegt mit der Geschwindigkeit $v$ durch zwei horizontale Kondensatorplatten mit der Länge L und wird dabei abgelenkt. 
\begin{equation}
\Delta z = \frac{a \cdot t^2}{2} = \frac{q \cdot E}{m}\frac{L^2}{2v^2} = \frac{E\cdot L^2}{2v^2} \cdot \frac{q}{m}
\end{equation}
Moleküle besitzen elektrische Dipolmomente aufgrund der Verschiebung von Ladungsschwerpunkten. Sie richten sich daher in einem elektrischen Feld aus. Atome selbst besitzen kein Dipolmoment, es wird aber eines im elektrischen Feld induziert. \\
Im elektrischen Feld gibt es keine geschlossenen Feldlinien, es gilt also
\begin{equation}
\oint{\vec{E}\cdot d\vec{s}} \Rightarrow rot \ \vec{E} = 0
\end{equation}
\begin{center}
\textit{Das \textbf{elektrische Feld} ist \textbf{wirbelfrei} und somit ist $\vec{E}$ ein \textbf{konservatives Potential V} zuordenbar.}
\end{center}

\subsection{Stationäre elektrische Ströme}
Den Transport von elektrischen Ladungen nennt man Strom
\begin{equation}
I = \frac{dQ}{dt} = \int{\vec{j} \cdot d\vec{A}}
\end{equation}
wobei $j$ die \textbf{elektrische Stromdichte} ist. Die Kontinuitätsgleichung besagt, dass elektrische Ladungen weder erzeugt noch zerstört werden können. \\
Als Ladungsträger kommen hauptsächlich Elektronen und positive sowie negative Ionen in Frage. Bei elektronischen Leitern sind es hauptsächlich Elektronen, bei Ionenleitern hauptsächlich Ionen und bei gemischten Leitern kommen beide vor. Die \textbf{technische Stromrichtung} ist definiert als die Flussrichtung der \textbf{positiven Ladungsträger}, der technische Fluß geht also immer von \textbf{Plus nach Minus}.
Ohne äußeres Feld bewegen sich Ladungsträger ungeordnet und haben eine mittlere Geschwindigkeit von 0. Beim Ladungstransport stoßen Ladungsträger zusammen und geben dabei Energie ab, entlang eines Leiters findet stets ein Spannungsabfall statt, der die nötige Feldstärke zum Transport zur Verfügung stellt. \\
Für die Leistung gilt
\begin{equation}
P = U \cdot I = \frac{U^2}{R} = I^2 \cdot R
\end{equation}

\subsubsection*{Stromquellen}
Diese basieren auf einer Trennung von positiven und negativen Ladungen, bei dieser Trennung muss Arbeit geleistet werden. Räumlich getrennte Ladungen haben eine Potentialdifferenz, verbindet man diese beiden so fließt ein Strom, der vom äußeren Widerstand bestimmt wird.

\subsection{Statische Magnetfelder}
Magnete haben \textbf{immer Dipolcharakter}. Im \textbf{homogenen Magnetfeld} haben magnetische Dipole keine resultierenden Kräfte. Im \textbf{inhomogenen Feld} wirkt ein \textbf{Drehmoment} nd auch eine Kraft $\vec{F} = -\vec{p}_m\cdot \vec{B}$. \\
Für die Wirkung zwischen zwei Permanentmagneten gilt
\begin{equation}
\vec{F} = \frac{1}{4 \pi \mu_0}\frac{p_1 \cdot p_2}{r^2}\cdot \vec{r}_e \text{ mit } \mu_0 = 4\pi \cdot 10^{-7} \frac{Vs}{Am}
\end{equation}
Dabei ist $p$ die sogenannte Polstärke und $\mu_0$ die \textbf{magnetische Permeabilitätskonstante}. \\
Alle \textbf{magnetischen Feldlinien} sind \textbf{geschlossen}, es gilt also $div \ \vec{B} = 0$ und der Fluss durch eine geschlossene Fläche ist Null. 
\begin{center}
\textit{Das \textbf{statische elektrische Feld} ist also \textbf{wirbelfrei}, während das \textbf{statische magnetische Feld} Wirbel besitzt.}
\end{center}
Um einen stromdurchflossenen Leiter bildet sich ein Magnetfeld (mit der Rechtsschraubregel). Ein zu einer Spule gewickelter Leiter erzeugt ein annähernd homogenes Magnetfeld im Inneren. Der magnetische Fluss ist gegeben als
\begin{equation}
\Phi_m = \int_{A}{\vec{B}\cdot d\vec{A}}
\end{equation}
Weiter gilt, dass die magnetische Erregung entlang eines Weges die Summe aller umschlossenen Ströme ist
\begin{equation}
\oint{\vec{H} \cdot d\vec{s}} = I
\end{equation}

\subsubsection*{Kräfte auf bewegte Ladungen im Magnetfeld}

Man beobachtet, dass die Kraft stets senkrecht zum Geschwindigkeitsvektor der bewegten Ladung und senkrecht zum Vektor des magnetischen Feldes wirkt, die Kraft nennt man die \textbf{Lorentz-Kraft}
\begin{equation}
\vec{F} = q\cdot (\vec{v} \times \vec{B})
\end{equation}
Werden Elektronen in einem Leiter geführt, so können sie sich nicht ganz frei bewegen, daher wird auf den Leiter selbst eine Kraft ausgeübt
\begin{equation}
\vec{F} = \int_{L_1}^{L_2}{(\vec{j} \times \vec{B}) \ dV}
\end{equation}

\pagebreak


\subsubsection*{HALL-Effekt}
Die auf die Ladungsträger wirkende Lorentz-Kraft verursacht eine Ablenkung der Ladungsträger senkrecht zum Leiter, wobei sich zwischen den Berandungen des Leiters die Hallspannung $U_H$ aufbaut. Die Kraft auf die Ladungsträger im zugehörigen elektrischen Feld $E_H$ kompensiert die Lorentz-Kraft. Dieser Effekt wird oft zur \textbf{Magnetfeldmessung} eingesetzt (HALL-Sonde). 

\subsubsection*{Zusammenhang zwischen elektrischem und magnetischem Feld}
Das Magnetfeld eines Stromes und die LORENTZ-Kraft lassens ich aus dem COULOMB-Gesetz herleiten. Das Magnetfeld ist eine Änderung des elektrischen Feldes bewegter Ladungen, daher auch \textbf{elektromagnetisches Feld} genannt. 
\begin{table}[h!]
  \begin{center}
    \begin{tabular}{| r |  l |}
    \hline
    elektrisches Feld & magnetisches Feld  \\ \hline \hline
    $rot \ \vec{E} = 0$ & 	$rot \ \vec{B} = \mu_0 \cdot \vec{j}$  \\ \hline
    $div \ \vec{E} = \rho / \epsilon_0$ & $div \ \vec{B} = 0$  \\ \hline
    $\vec{E} = -grad \ \Phi$ & $\vec{B} = rot \ \vec{A}$ \\ \hline
    $\vec{j} = \sigma \cdot \vec{E}$ & $\epsilon_0 \cdot \mu_0 = 1 / c^2$ \\
    \hline
    \end{tabular}
  \end{center}
  \caption{Zusammenhang der bisher verarbeiteten wichtigen Vektorgrößen}
\end{table}

\subsubsection*{Materie in magnetischen Feldern}

Es gilt allgemein
\begin{equation}
B_{Materie} = \mu \cdot B_{Vakuum} = \mu \cdot \mu_0 \cdot H_{Vakuum}
\end{equation}
Dabei nennt man $\mu$ die \textbf{relative Permeabilität} und ähnlich wie beim elektrischen Feld beobachtet man eine magnetische Polarisierung der Materie. \\
\textbf{Diamagnetische Stoffe} besitzen kein permamentes magnetisches Dipolmoment, die induzierten Dipole entstehen so, dass sie dem äußeren Feld entgegengesetzt orientiert sind. \\
\textbf{Paramagnetische Stoffe} besitzen permanente magnetische Dipolmomente, die aber statistisch verteilt sind, dass äußere Feld richtet die magnetischen Dipole teilweise aus, wodurch dir Magnetisierung stark zunimmt. 
\\
\textbf{Ferromagnetische Stoffe} haben permanentes magnetisches Verhalten, es entstehen sogenannte WEISS'sche Bezirke, in denen sich magnetische Dipole parallel ausrichten. Nach Wegnahme des Feldes richten sich die Orientierungen der einzelnen Bezirke wieder statistisch aus, die Magnetisierung hängt von der Vorgeschichte ab (\textbf{Hysterese}).

\pagebreak 

\subsection{Zeitlich veränderliche elektrische und magnetische Felder}
\subsubsection*{Induktion}
Wenn sich ein Leiter in einem magnetischen Feld bewegt, wird eine Spannung induziert. \begin{equation}
U_{ind} = -\frac{d\Phi_m}{dt}
\end{equation}
Der Spannungsstoß ist bei langsamer bzw. schneller Durchführung gleich groß.
\begin{center}
\textit{Ein \textbf{zeitlich veränderndes magnetisches Feld} erzeugt ein \textbf{elektrisches Wirbelfeld} mit \textbf{geschlossenen Feldlinien}. Es gilt ganz allgemein $rot \ \vec{E} = -\frac{d\vec{B}}{dt}$}
\end{center}

\begin{table}[h!]
  \begin{center}
    \begin{tabular}{| c |  c | c |}
    \hline
    Art des elektrischen Feldes & elektrostatisch & Wirbelfeld  \\ \hline \hline
    Quelle & 	Ladungen & erzeugt durch $d\vec{B}/dt$ \\ \hline
    Feldlinien & offen $\Rightarrow$ $rot \ \vec{E} = 0$ & geschlossen $rot \ \vec{E} = -d\vec{B}/dt$    \\ \hline
    Potential & konservativ $\Rightarrow \vec{E} = -grad \ \Phi$ & nicht darstellbar als Gradient \\ \hline
    \end{tabular}
  \end{center}
  \caption{Gegenüberstellung}
\end{table}
 Es ist also völlig verschieden vom statischen elektrischen Feld, dessen Feldlinien offen sind. 

\subsubsection*{Selbstinduktion}
Dies bedeutet, dass eine der erzeugenden Spannung entgegengesetzte induzierte Spannung entsteht, dies hemmt das Ansteigen des Stromes und erzeugt beim Abschalten eine Spitzenspannung. \\
Die im magnetischen Feld enthaltene Energie wird berechnet durch
\begin{equation}
W_{magn} = \frac{I^2\cdot L}{2}
\end{equation}

\begin{table}[h!]
  \begin{center}
    \begin{tabular}{| c | c |}
    \hline
    Elektrisches Feld & Magnetisches Feld  \\ \hline \hline
    $W_{el} = \frac{1}{2} C \cdot U^2$ & $W_{magn} = \frac{1}{2} L \cdot I^2$ \\ \hline
    $u_{el} = \frac{1}{2}\epsilon_0 E^2$ & $u_{magn} = \frac{1}{2}\mu_0 \cdot H^2 = \frac{1}{2\mu_0}B^2$ \\ \hline
    \end{tabular}
  \end{center}
  \caption{Gegenüberstellung}
\end{table}

\subsubsection*{Verschiebungsstrom}
Dieser wird definiert als das Verschieben von Ladungen über (nicht verbundene) Kondensatorplatten. Magnetfelder werden als nicht nur von Strömen erzeugt, sondern auch von sich ändernden elektrischen Feldern. 

\pagebreak

\subsection{Elektromagnetische Schwingungen und Wellen}

Eine Schaltung aus Kondensator $C$ und Spule $L$ bildet einen \textbf{elektromagnetischen Schwingkreis}. $C$ sei mit $W_{el} = \frac{C \cdot U^2}{2}$ geladen. Beim Entladen lädt sich die Spule mit $W_{magn} = \frac{L \cdot I^2}{2}$ auf. Ist ein Widerstand $R$ vorhanden, erfolgt die Schwingung gedämpft. \\
Es gibt 3 Arten, wie ein Schwingkreis schwingen kann

\begin{enumerate}
\item Im \textbf{Schwingfall} handelt es sich um eine echte gedämpfte Schwingung, beim \textbf{aperiodischen Grenzfall} schwingt der Kreis einmal und stoppt danach und beim \textbf{Kriechfall} wird langsam eine Amplitude aufgebaut, die sich aber gleich mit der Dämpfung wieder Null nähert. 
\item Bei der \textbf{erzwungenen Schwingung} wird der Schwingkreis von außen mit der Frequenz $\omega$ angeregt. Im Resonanzfall wird die Schwingungsamplitude hierbei sehr groß. \item Eine \textbf{ungedämpfte elektrische Schwingung} kann durch Zuführung der verlorenen Energie über eine Rückkopplung erreicht werden. Die MEISSNER-Schaltung war eine der ersten brauchbaren Schaltungen dieser Art.
\end{enumerate}
Die Informationsübertragung über elektromagnetische Wellen erfolgt durch Modulation des Signals. Bei der \textbf{Stabantenne} handelt es sich um einen offenen Schwingkreis, der elektromagnetische Wellen abstrahlt. Die optimale Länge der Antenne ist $l = \lambda/2$ und die abgestrahlte Leistung entspricht ungefähr $\omega^4$. Bei der Ausbreitung der Wellen wechseln sich ständig elektrische und magnetische Energie ab $\vec{E} \bot \vec{B}$. 

\pagebreak

\section{Optik}
\subsection{Elektromagnetische Wellen}

Elektromagnetische Wellen bestehen aus \textbf{6 Komponenten}, $E_{x,y,z}$ des elektrischen Feldes und $B_{x,y,z}$ des magnetischen Feldes. Die \textbf{Ausbreitungsgeschwindigkeit} einer Welle ist mit $v = \frac{1}{\sqrt{\epsilon_0 \cdot \mu_o}}$ gegeben, im Vakuum beträgt diese $c \approx 3\cdot 10^8 \ m/s$, Licht ist eine \textbf{elektromagnetische Transversalwelle}. \\
Die \textbf{Energiedichte}[$J/m^3$] ist der Energieinhalt des elektromagnetischen Feldes pro Volumseinheit, die gesamte Energiedichte ist gegeben durch

\begin{equation}
u = u_E + u_B =\epsilon_0 E^2 = \frac{1}{\mu_0}B^2 = \sqrt{\frac{\epsilon_0}{\mu_0}}E \cdot B
\end{equation}
und entspricht dem \textbf{Strahlungsdruck p'}. Die \textbf{Photonenenergie} ist gegeben durch
\begin{equation}
E_{Photon} = h \cdot \nu \text{ mit } h = 6.6 \cdot 10^{-34} [Js]
\end{equation}
und der \textbf{Impuls des Photons} ergibt sich zu 
\begin{equation}
p_{Photon} = \frac{E_{Photon}}{c} = \frac{h\nu}{c}
\end{equation}
Aus der Wellenlehre ist bekannt
\begin{equation}
\lambda\nu = c
\end{equation}
\subsubsection*{Abstrahlungsvorgänge}
Ursache für die Aussendung elektromagnetischer Strahlung ist eine ungleichförmig bewegte Ladung z.B.: linear beschleunigte Ladung. Bei der \textit{Dipolstrahlung} werden Elektronen im Inneren einer Stabantenne periodisch verschoben, es kommt dabei zu einer \textbf{Abstrahlung}. 
Bei der \textit{Abstrahlung von Energie durch Atome} handelt es sich um den Übergang zwischen Energieniveaus, die Energiedifferenz wird abgestrahlt
\begin{equation}
E_{Photon} = h\nu = E_2-E_1
\end{equation}
Auch \textit{heiße Körper} strahlen Energie ab, Ursache ist die Temperaturbewegung der Körperteilchen (in allen Frequenzen = \textbf{Strahlungskontinuum}).

\subsection{Ausbreitung von Licht}
\subsubsection*{Dispersion}
Im Vakuum breitet sich Licht als ebene Welle oder als Kugelwelle mit \textbf{Phasengeschwindigkeit c} aus. In Materie kommt zu $\epsilon$ und $\mu$ noch jeweils ein Relativitätsfaktor dazu vlg. $\epsilon = \epsilon_r \cdot \epsilon_0$. Diese Faktoren sind bei der Berechnung der Phasengeschwindigkeit zu berücksichtigen. 
Der \textbf{absolute Brechungsindex} wird angegeben als
\begin{equation}
n = \frac{c}{\nu} = \sqrt{\frac{\epsilon\cdot\mu}{\epsilon_0\cdot\mu_0}} = \sqrt{\epsilon_r \cdot \mu_r} \approx \sqrt{\epsilon_r} \text{ da } \mu_r \text{ sehr klein ist}
\end{equation}
Die Brechzahl n ist frequenzabhängig $n = n(\nu)$, dies nennt man \textbf{Dispersion}.  
\pagebreak
\\
Bei der Brechung findet eine \textbf{Polarisation} der Materie statt, es gibt hierbei Ionen- und Elektronenpolarisation. Ionen sind schwerer und folgen daher schwerer den Änderungen des elektromagnetischen Feldes, Elektronen dagegen sogar bei sehr hohen Frequenzen. Die Brechzahl steigt mit wachsender Frequenz und abnehmender Wellenlänge. An Resonanzstellen zeigt sich aber ein umgekehrtes Bild. \\
Die \textbf{Phasengeschwindigkeit} wird angegeben mit 
\begin{equation}
\nu = \frac{c}{n}
\end{equation}
\subsubsection*{Fortpflanzung des Lichts in Materie}
Die Welle trifft auf Atome, deren Elektronen mitschwingen und damit der Welle Energie entziehen und gleichzeitig strahlen die Elektronen Energie ab wodurch \textbf{Sekundärwellen} entstehen, welcher der Phase der Primärwellen vor- oder nacheilen können. Normalerweise ist die Sekundärwelle schwächer und wird absorbiert, \textbf{Absorption und Dispersion} sind gleichzeitig vorhanden. 
\subsubsection*{Reflexion und Brechung}
\begin{center}
\textit{Jeder Punkt auf einer \textbf{primären Wellenfront} ist Ausgangspunkt einer Kugelwelle (Sekundärwelle). Die Wellenfront zu einem späteren Zeitpunkt ist die Einhüllende der \textbf{sekundären Elementarwellen}.}
\end{center}

Bei der Reflexion gelten folgende Gesetze

\begin{itemize}
\item Einfallender, reflektierter und gebrochener Strahl liegen in der Einfallsebene (definiert durch den einfallenden Strahl und die Flächennormale)
\item Einfallswinkel = Reflexionswinkel $\Rightarrow \theta_i = \theta_r$
\item SNELLIUS'sche Brechungsgesetz: $\frac{sin \ \theta_i}{sin \ \theta_t} = \frac{n_t}{n_i}$
\begin{itemize}
\item Von optisch dichtem ins optisch dünne Material wird \textbf{vom Lot gebrochen}
\item Vom optisch dünnen ins optisch dicke Material wird \textbf{zum Lot gebrochen}
\item \textbf{Totalreflexion} tritt dann auf, wenn der Grenzwinkel überschritten wird \\($\theta_i > arcsin \ \frac{n_t}{n_i}$)
\end{itemize}
\end{itemize}
\textbf{FERMAT'sche Prinzip}: Das Licht durchläuft stets die kürzeste (optische) Strecke zwischen zwei Punkten. Die Zeit für den Weg nimmt einen Maximalwert an. Die optische Wellenlänge wird definiert durch $s = n \cdot d$ mit n als Brechzahl und d als Weglänge. \\
Der Winkel, bei dem keine Reflexion stattfindet, wird \textbf{BREWSTER- Winkel genannt}
\begin{equation}
\theta_{iB} = \text{arctan } \frac{n_t}{n_i}	
\end{equation}

\pagebreak

\subsubsection*{Streuung}
Die Streuung ist die \textbf{Sekundärstrahlung}, welche zu einem kleinen Teil in alle Raumrichtungen mit der selben Frequenz und Phase wie die Hauptwelle gestreut wird (kohärente Streuung). Die Flußdichte S ist frequenzabhängig und proportional zu $\nu^4$. \textbf{Deswegen wird auch blaues Licht stärker gestreut als rotes Licht (Himmelsblau, Morgen- und Abendrot)}.


\subsection{Geometrische Optik}
Eine reale Welle wird beim Durchgang durch eine Spalt gebeugt, besonders stark, wenn die Abmessungen der beugenden Strukturen mit der Wellenlänge vergleichbar sind. Wird dabei ein Objektpunkt in exakt einen Bildpunkt umgewandelt, so spricht man von einer \textbf{stigmatischen Abbildung}. \textbf{Reale optische Systeme} haben \textbf{Abbildungsfehler} aufgrund von \textbf{geometrischer Unvollkommenheiten} und \textbf{Beugungseffekten}. 
\subsubsection*{Linsen}
Sie bestehen im Allgemeinen aus rotationssymmetrischen Glaskörpern mit sphärischen Grenzflächen. Zur Berechnung von Linsen
\begin{equation}
 \text{Abbildungsgleichung: } \frac{1}{g} + \frac{1}{b} = \frac{1}{f} \text{ und die Größe: }M_T = -\frac{b}{g}
 \end{equation} 
 \begin{itemize}
 \item g = f $\Rightarrow$ b = $\infty$
 \item g = $\infty$ $\Rightarrow$ b = f
 \item f = 2f $\Rightarrow$ b = 2f
 \end{itemize}
 
Achsenferne Punkte werden konstruiert, indem man einen Mittelpunktstrahl, einen Strahl parallel zur Achse und einen Strahl durch den Brennpunkt zeichnet. 
 
\subsubsection*{Spiegel}
Ebene Spiegel erzeugen ein \textbf{virtuelles, aufrechtes, seitenverkehrtes Bild}. Gekrümmte Spiegel haben einen Brennpunkt, in dem sich alle parallel einfallenden Strahlen schneiden. 
\begin{equation}
 \text{Abbildungsgleichung: } \frac{1}{g} + \frac{1}{b} = \frac{1}{f} = -\frac{2}{R}
 \end{equation}
\subsubsection*{Prismen}
Man unterscheidet \textbf{Dispersionsprismen} (Zerlegung eines einfallenden Lichtbündels in verschiedene Wellenlängen) und \textbf{Reflexionsprismen} (Ab- bzw. Umlenkung von Strahlenbündeln). 

\pagebreak

\subsection{Optische Instrumente}

\subsubsection*{Das Auge - Fehlsichtigkeit}
Die Brennweite des Auges ist ca. 17mm, das Auflösungsvermögen $\alpha_{min} = 1'$. Beim \textbf{fehlsichtigen} Auge ist der Brennpunkt des entspannten Auges bei \textbf{Kurzsichtigkeit vor} und bei \textbf{Weitsichtigkeit hinter} der Netzhaut. Als \textbf{Astigmatismus} bezeichnet man unterschiedliche Brecheigenschaften des Auges für waagrechte und senkrechte Ebenen (Korrektur mit Zylinderlinse). Die Brechkraft von Linsen wird in Dioptrien angegeben
\begin{equation}
D = \frac{1}{f}
\end{equation}

\subsubsection*{Lupe}
Diese bewirken eine \textbf{Vergrößerung des Sehwinkels}. 
\begin{equation}
V = \frac{d_0}{f} \text{ mit } d_0 \text{ = 25cm, deutliche Sehweite (per Definition)}
\end{equation}

\subsubsection*{Mikroskop}
Bestehen aus zwei Linsen, das Objektiv erzeugt dabei ein vergrößertes, umgekehrtes Bild und das Okular wirkt als Lupe für das Objektivbild, das Bild liegt in der Brennweite des Okulars.
\begin{equation}
V_{ges} = V_O \cdot V_{Ok} = \frac{L = 160mm}{f_O}\cdot \frac{d_0 = 250mm}{f_{Ok}} 
\end{equation}

\subsubsection*{Fernrohr}
Es gibt Refraktoren mit Linsen und Spiegelteleskope. Das astronomische Fernrohr funktioniert wie ein Mikroskop, aber für $\infty$ weit entfernte Gegenstände. Da das Bild verkehrt ist, wird im Feldstecher ein Aufrichtesystem eingesetzt. Beim GALILEI-Fernrohr wird beim Okular eine Zerstreuungslinse eingesetzt. 

\subsubsection*{Kamera}
Ziel ist die Abbildung von im Vergleich zur Brennweite meist sehr weit entfernten Objekten. Anforderungen an das Objektiv sind unter anderem in großes Öffnungsverhältnis, chromatische Korrektur, Bildfeldebnung, keine Verzerrung und Entspiegelung der Oberfläche. 

\subsubsection*{Diaprojektor}
Hohlspiegel - Licht - Kondensatorlinse - Dia - Objektiv - Leinwand

\subsubsection*{Spektrograph}
Beim \textbf{Prismenspektrograph} ist das dispergierende Element ein Glasprisma. Beim \textbf{Gitterspektrograph} wird die Reflexion am Beugungsgitter statt des Prismas verwendet, es werden dabei erst Beugungen erster oder höherer Ordnung verwendet. 

\subsubsection*{Abbildungsfehler}
\begin{itemize}
\item \textbf{Sphärische Abberation}: Parallele Strahlen werden nicht im Brennpunkt gebündelt
\item \textbf{Astigmatismus}: Die Brennpunkte horizontaler und vertikaler Linien fallen nicht zusammen, beispielsweise bei Zylinderlinsen, sphärischen Linsen und nicht auf der optischen Achse liegenden Objektpunkten
\item \textbf{Bildfeldkrümmung}: Bild liegt auf einer gekrümmten Fläche, nicht einer Ebene
\item \textbf{Verzerrung}: Es kommt zu verzerrten Objektdarstellungen, Korrektur durch Mehrlinsensysteme
\item \textbf{Chromatische Abberation}: Die Lichtfarben bündeln sich nicht im Brennpunkt eines Objektivs und führen zu Farbsäumen im Bild, Korrektur mittels Zerstreuungslinse. 
\end{itemize}

\subsection{Polarisation}

Die Polarisierbarkeit des Lichtes ist nur erklärbar, wenn Licht als \textbf{transversale elektromagnetische Welle} aufgefasst werden kann.  So eine transversale Welle in Richtung z kann beschrieben werden durch
\begin{equation}
\vec{E} = \vec{E}_0 \cdot \text{cos}(k\cdot z - \omega \cdot t)
\end{equation}
$E_0$ kann dabei in zwei Komponenten zerlegt werden (x- und y-Achse), wenn eine der beiden Teilwellen eine Phasendifferenz gegenüber der anderen Teilwelle aufweist, wird sich die räumliche Lage von $\vec{E}_0$ ändern. 
\subsubsection*{Linear polarisierte Welle}
Ist die Phasendifferenz zwischen den Teilwellen ein \textbf{ganzzahliges Vielfaches von $\pi$}, so erhält man eine \textbf{linear polarisierte Welle}. 

\subsubsection*{Zirkular polarisierte Welle}
Ist die Phasendifferenz zwischen den Teilwellen ein \textbf{ganzzahliges Vielfaches von $\pi /2$} und sind die \textbf{Amplituden der Teilwellen gleich groß}, so erhält man eine \textbf{zirkular polarisierte Welle}. Rechtszirkular polarisiertes Licht bedeutet, dass die Spitze des Feldstärkenvektors in Ausbreitungsrichtung eine Rechtsschraube beschreibt. 

\subsubsection*{Allgemeiner Fall: Elliptisch polarisierte Welle}
Im Allgemeinen werden die Amplituden der Teilwellen verschieden groß sein und auch die Phasendifferenzen nicht in die vorherigen Kategorien fallen. Dies nennt man eine \textbf{elliptisch polarisierte Welle}. Im natürlichen Licht sind viele linear polarisierten Wellenzüge vorhanden, das Licht erscheint dadurch \textbf{unpolarisiert}. 

\subsubsection{Polarisatoren}
Diese erzeugen aus natürlichem Licht linear polarisiertes Licht.
\begin{itemize}
\item \textbf{Dichroismus}: Selektive Absorption von schwingenden Komponenten (z.B.: Drahtgitterpolarisator für Mikrowellen). Elektrischer Vektor parallel zu den Drähten $\Rightarrow$ wird zum Schwingen angeregt, \textbf{Welle stark absorbiert}. Elektrischer Vektor senkrecht zu Drähten $\Rightarrow$ keine Absorption. 
\item \textbf{Doppelbrechung}: In manchen Kristallen sind die Lichtwege für unterschiedliche Schwingungsrichtungen anders - optisch \textbf{anisotrop}.
\item \textbf{Reflexion}: Parallel zur Einfallsebene polarisiertes Licht wird transmittiert, normal zur Einfallsebene polarisiertes Licht wird reflektiert.
\end{itemize}

\subsubsection{Erzeugung von Phasenverschiebungen}
Schneidet man doppelbrechende Kristalle parallel zur optischen Achse, nehmen ordentliche und außerordentliche Strahlen denselben geometrischen Weg, besitzen aber verschieden Ausbreitungsgeschwindigkeiten $ \Rightarrow$ Phasendifferenz. \\
Mit einem Plättchen, das eine Phasendifferenz von $\Delta \varphi = k\cdot 2\pi + \pi/2$ erzeugt, wird eine Verschiebung der Wellenzüge um $\lambda/4$ erzeugt $\Rightarrow$ \textbf{zirkular polarisiertes Licht}. \\
Mit einem Plättchen, das eine Phasendifferenz von $\Delta \varphi = k \cdot \pi$ erzeugt, kann mit $\lambda/2$ die Polarisationsrichtung der Welle gedreht werden. 

\subsubsection{Optische Aktivität}
Damit bezeichnet man die Drehung der Polarisationsebene einer fortschreitenden Welle bei Durchgang durch ein Medium, optisch aktive Moleküle können als schraubenförmige "Leiter" angesehen werden, es kommt zu einer Drehung der Polarisationsebene im Schraubensinn. 

\subsubsection{Erzwungene optische Effekte}
\begin{itemize}
\item \textbf{Spaltdoppelbrechung}: Bei dieser wird in einem isotrpen Medium durch mechanische Spannung Doppelbrechung induziert und wird dadurch anisotrop, im Allgemeinen entsteht aus linear polarisierten Licht elliptisch polarisiertes Licht. 
\item \textbf{FARADAY-Effekt}: Drehung der Polarisationsebene von Licht in ansonst isotoper Materie, wenn ein Magnetfeld in Richtung der Ausbreitungsrichtung angelegt wird. Allerdings kehrt sich der Drehsinn um, wenn das Licht in die andere Richtung läuft. Wird angewendet als \textbf{FARADAY-Isolator}, welcher in der Lasertechnik verhindert, dass reflektierte Laserstrahlung in den Laser eintritt und als \textbf{FARADAY-Modulator}, zur Modulation der Lichtintensität, beispielsweise bei der Signalübertragung. 
\item \textbf{KERR-Effekt}: Durch ein elektrisches Feld wird ein optisch isotopes Medium doppelbrechend, die optische Achse ist gleich der Richtung des elektrischen Feldes. Gut geeignet sind Flüssigkeiten, deren Moleküle ein permanentes elektrisches Dipolmoment besitzen. Im elektrischen Feld werden Moleküle parallel ausgerichtet. Andwendung: Zum "Schalten" von Licht bis 10 GHz.
\item \textbf{POCKELS-Effekt}: In bestimmten Kristallen wird durch ein elektrisches Feld Doppelbrechung induziert, die linear zur Feldstärke ist. Wie KERR, allerdings mit linearem Zusammenhang und bis 25 GHz. 
\end{itemize}

\subsection{Interferenz}
Allgemein gilt das Superpositionsprinzip
\begin{equation}
\vec{E} = \vec{E}_1 + \vec{E}_2
\end{equation}
Die Intensität ergibt sich zu 
\begin{equation}
I = c \cdot \epsilon_0 \cdot \langle E^2 \rangle
\end{equation}
Bei Überlagerung kommt es zu Interferenzen, konstruktive Interferenz ist gegeben als
\begin{equation}
I_1 = I_2 \ \Rightarrow \ I_{ges} = 4 \cdot I_1
\end{equation}
Bei destruktiver Interferenz gilt $I_{ges} = 0$. 
\subsubsection*{Kohärenzzeit und Kohärenzlänge}
Kohärenz ist die Eigenschaft von Wellen bei Überlagerung Interferenzmuster zu zeigen. Wellenzüge haben nur eine endliche Länge, die mit der Abstrahlungszeit zusammenhängt, es gilt hierbei 
\begin{equation}
l = c \cdot \Delta t
\end{equation}
Die Zeit $\Delta t $, die eine Welle mit \textbf{unveränderter Phase} schwingt nennt man die \textbf{Kohärenzzeit}, die dazugehörige Länge $l_c = c \cdot \Delta t$ \textbf{Kohärenzlänge}. Dies ist messbar mit einem Interferometer, welches Licht aufteilt und auf zwei unterschiedlich langen Wegen leitet und danach interferieren lässt (Michelson-Interferometer). 
Eine große Bandbreite der Frequenzen schränkt die Kohärenzlänge ein, die Kohärenzlänge von weißem Licht ist sehr klein ($\approx 2\mu m$), während monochromatisches Licht eine viel größere Kohärenzlänge ($\approx 10 cm$) und Laserlicht sogar bis zu 100 m. All das wird unter \textbf{zeitlicher Kohärenz} zusammengefasst.
\\ Die \textbf{räumliche Kohärenz} befasst sich mit Phänomenen, die von der räumlichen Ausdehnung der Lichtquelle herrühren, zum Beispiel die kohärente Beleuchtung eines Doppelspaltes. 
\subsubsection*{Interferometer mit Amplitudenaufspaltung}
Es werden Teilwellen mittels eines Strahlenteilers aus einer Primärwelle erzeugt, diese Teilwellen werden wieder vereinigt an einem zweiten oder auch demselben Strahlenteiler, die Phasendifferenz ist dabei vom Weg abhängig. 
\\
Bei \textbf{weißem Licht} handelt es sich um eine Überlagerung von Teilwellen, um Interferenz beobachten zu können, muss die Phasendifferenz für alle Teilwellen in einem Punkt aufgehoben sein, es sind daher nur wenige Interferenzstreifen sichtbar. \\
An \textbf{planparallelen Schichten} wird ein Strahl von der Oberfläche sowie an der Unterseite reflektiert, es treten parallele Strahlen aus, die durch eine Linse zur Interferenz gebracht werden können. Bei einem bestimmen Winkel tritt \textbf{konstruktive Interferenz} auf, dies nennt man auch \textbf{Interferenzen gleicher Neigung}. \\
Bei \textbf{Interferenz gleicher Dicke} wird die Phasendifferenz durch die unterschiedliche Dicke der Schicht bestimmt. 

\subsubsection*{Reflexionsmindernde Schichten}
\textbf{Entspiegelung} durch eine $\lambda/4$-dicke Schicht, wo der 1. und 2. reflektierte Strahlt destruktiv interferieren, dabei muss allerdings die Amplitude der Teilwellen gleich groß sein. \\
Bei \textbf{Mehrfachschichten} kann eine wirksamere Entspiegelung erreicht werden, die Entspiegelung vergrößert dabei die Intensität der transmittierten Welle.

\subsubsection*{Zweistrahlinterferenz}
Die auf das Interferometer einfallende Welle wird in \textbf{zwei Teilwellen} mit \textbf{annähernd gleicher Amplitude} aufgespalten. Beim MICHELSON-Interferometer bilden zwei Spiegel eine virtuelle planparallele Platte, es wird dabei ein Interferenzmuster in Form von Ringen erzeugt. Es wurde damit der Lichtäther falsifiziert und es kann genutzt werden zur Wellenlängenmessung, zur Längenmessung und zur FOURIER-Transformations-Spektroskopie.  \\
Beim SAGNAC-Interferometer wird das Licht über 4 Wege im Kreis geführt, während der gesamte Versuchsaufbau rotiert. Es kann verwendet werden zur Messung von Winkelgeschwindigkeiten und als Lasergyroskop. 

\subsubsection*{Vielstrahlinterferenz}
Die einfallend Welle wird in viele Teilwellen mit abnehmender Amplitude aufgespalten. Beim FABRY-PEROT-Interferometer werden diese Strahlen mit einer Linse gebündelt und erzeugen ein \textbf{Ringsystem konzentrisch um die optische Achse}. Anwendung findet es in hochauflösenden Interferenzspektroskopen, als Laserresonator oder zur Frequenzselektion in der Laserphysik. 


\subsection{Beugung}

Eigentlich besteht kein prinzipieller Unterschied zwischen \textbf{Beugung} und \textbf{Interferenz}, man spricht von Interferenz, wenn sich im allgemeinen relativ wenige Wellen (2 bis einige 100) überlagern und von Beugung, wenn sich sehr viele Wellen (viele zehntausende) überlagern. \\
Unter Beugung versteht man die Abweichungen von der geradlinigen Ausbreitung, wenn die freie Ausbreitung durch Hindernisse gestört wird. \\
\textbf{FRAUENHOFERsche Beugung}: Beugungserscheinungenin unendlich großer Entfernung vom beugenden Objekt (\textbf{Fernfeld}). z.B.: werden bei einer Linse parallele Strahlen gesammelt und in die Brennebene transformiert. \\
\textbf{FRESNELsche Beugung}: Erscheinung in der Nähe des beugenden Objekts (\textbf{Nahfeld}).

\pagebreak

\subsubsection{Beugung am Spalt}

Jeder Punkt des Spalts ist eine Sekundärwelle, die miteinander interferieren. Bei $\theta = 0$ sind alle Sekundärwellen in Phase, es kommt zu konstruktiver Interferenz und man erhält das \textbf{zentrale Maximum}. \\
Für das erste Beugungsminimum gilt $sin \ \theta_1 = \pm \frac{\lambda}{b}$. Der von der Mitte ausgehende Strahl und der Randstrahl interferieren destruktiv. 
\begin{table}[h!]
   \begin{center}
     \begin{tabular}{| c | c |}
     \hline
     $\theta = 0$ & größte Intensität  \\ \hline 
     $sin \ \theta = \frac{\lambda}{2b}$ & halbe Intensität \\ \hline
     $sin \ \theta = \frac{\lambda}{b}$ & 1. Minimum \\ \hline
     $sin \ \theta = \frac{3 \lambda}{2b}$ & 1. Nebenmaximum \\ \hline
     $sin \ \theta = \frac{2 \lambda}{b}$ & 2. Minimum \\ \hline     
     \end{tabular}
   \end{center}
   \caption{Spezialfälle}
 \end{table} 
 \\
Man beobachtet also \textbf{Minima} (Auslöschung) für 
\begin{equation}
sin \ \theta = k \cdot \frac{\lambda}{b} \text{ mit } k = 1,2,3...
\end{equation}
Bei der \textbf{Beugung am Doppelspalt} gilt für die Maxima der Interferenz (mit a = Spaltabstand)
\begin{equation}
sin \ \theta = k \cdot \frac{\lambda}{a} \text{ mit } k = 1,2,3...
\end{equation}
Hierbei ist die Spaltbreite vernachlässigbar, wird sie dennoch berücksichtigt, wird die zuvor definierte Interferenzstruktur mit der Intensitätsverteilung der Beugung am Einzelspalt moduliert. \\
Bei der \textbf{Gitterbeugung} interferieren sehr viele Teilwellen und es bilden sich deutliche Maxima.

\subsubsection{Auflösungsvermögen optischer Systeme}

Jeder Objektpunkt wird durch ein optisches Element endlicher Öffnung abgebildet. Jeder Bildpunkt hat also mindesten den Durchmesser des zugehörigen Beugungsscheibchens. Als noch unterscheidbar gelten zwei Beugungsscheibchen, wenn die den halben Durchmesser auseinander liegen. Die minimale Auflösung des abbildenden Systems ist 
\begin{equation}
\Delta \varphi_{min} = 1.22 \frac{\lambda}{D} \text{ mit D = 2R (Objektiv-Durchmesser)}
\end{equation}

\pagebreak

\section{Quantennatur des Lichtes und der Materie}
\subsection{Quantennatur des Lichtes}
Es gab zwei Gleichungen zur Beschreibung von Strahlung eines schwarzen Körpers, die \textbf{RAYLEIGH-JEANS-Formel} für große Wellenlängen und die \textbf{WIEN-Formel} für kleine Wellenlängen. Diese wurden allerdings abgelöst durch die Formel von PLANCK, die den gesamten Wellenlängenbereich beschreiben konnte. Es gilt dabei
\begin{center}
\textit{Die Energie eines Oszillators, der mit der Frequenz $\nu$ schwingt, ist ein \textbf{ganzzahliges Vielfaches} der "Energieportion" $h \cdot \nu$}
\end{center}

Die Hypothese von gequantelten Energieportionen stand aber im Widerspruch zur elektromagnetischen Theorie des Lichts. 
\subsubsection{Photoelektrischer Effekt}
Ultraviolettes Licht trifft auf eine Elektrode und löst dort Elektronen aus. Dabei muss die vom Licht bereitgestellte Energie die Elektronen aus der Metalloberfläche der Elektrode befreien. Die hierzu nicht verbrauchte Energie übernehmen die Elektronen in Form von kinetischer Energie. Diese kann man nun gegen ein Potential anlaufen lassen, um die Energie zu messen. Solange trotz einer angelegten Gegenspannung Elektronen auf der Elektrode ankommen, fließt noch ein Strom. Wird der Strom zu Null, kann man über die anliegende Gegenspannung die kinetische Energie bestimmen
\begin{equation}
E_{kin} = \frac{m_e \cdot v^2}{2} = e \cdot U_0
\end{equation}
Die \textbf{kinetische Energie} der Elektronen ist \textbf{eine Funktion der Frequenz}, aber \textbf{keine Funktion der Intensität}. \\
 Umgekehrt ist der gemessene Strom (bei festgehaltener Gegenspannung) eine \textbf{Funktion der Strahlungsintensität}, aber \textbf{nicht abhängig von der Frequenz}. \\
EINSTEIN entwickelte daraus seine berühmte Formel
\begin{equation}
h \cdot \nu = \frac{m_e \cdot v^2}{2} + W \text{ mit W ... Ablösearbeit}
\end{equation}

\begin{center}
\textit{Die Energie $h \cdot \nu$ eines Lichtquants ist gleich der kinetischen Energie der Elektronen abzüglich der Ablösearbeit.}
\end{center}
Daraus lässt sich folgern
\begin{itemize}
\item \textbf{Lichtquanten} besitzen diskrete Energie
\item Energie wird vollständig auf Elektronen übertragen
\item Anzahl der ausgelösten Elektronen ist \textbf{unabhängig} von der \textbf{Wellenlänge}, aber \textbf{proportional} zur \textbf{Anzahl der auftreffenden Lichtquanten}
\item Lichtquant verhält sich wie ein stoßendes Teilchen $\Rightarrow$ Rückkehr zum \textbf{Korpuskelbild}
\end{itemize}


\pagebreak

\subsubsection{COMPTON-Effekt}
Untersucht wird das Streuspektrum eines Kristallspektrometers von Röntgenlicht nach einer Wechselwirkung mit freien Elektronen einer Probe (z.B.: Graphit). Man erkennt in den Spektren, dass eine vom \textbf{Winkel $\theta$ abhängige Verschiebung der Wellenlänge} auftritt. Ebenfalls detektiert man auch die unverschobene Wellenlänge (Streuung an Atomen). Diese Messergebnis lässt sich nur interpretieren, wenn man einen Stoß der Photonen mit freien Elektronen annimmt. Beim Stoß müssen dabei die Erhaltungssätze für Energie und Impuls gelten. Beim Ablenken am Graphit müssen also die Stoßgesetze gelten (Photonen stoßen mit freien Elektronen zusammen und dies führt zu einer Verschiebung der Wellenlänge, da die Elektronen Energie verlieren), es muss sich also bei den Photonen um Teilchen handeln. \\
Die Wellenlänge der gestreuten Röntgenstrahlung wird mit einem  Kristall-Röntgenspektrometer bestimmt , als Beugungsgitter werden die Kristallgitterebenen verwendet, man nutzt also zum \textbf{Nachweis der Teilchennatur} (Wellenlängenänderung) die \textbf{Welleneigenschaften} der Röntgenstrahlung aus. \\
Das Licht halt also \textbf{Wellen- und Teilcheneigenschaften}. 



\subsubsection{Erzeugung von Röntgenphotonen}

Im COMPTON-Experiment werden Röntgenphotonen als Teilchen identifiziert. Die Abstrahlung elektromagnetischer Wellen erfolgt durch beschleunigte elektrische Ladungen. Die Elektroden werden mit einer Spannung zur Anode beschleunigt und erreichen diese mit der Gesamtenergie
\begin{equation}
e \cdot U = \frac{m \cdot v^2}{2}
\end{equation}
Im Feld der Atomkerne der Anode werden die Elektronen verzögert und geben einen Teil der kinetischen Energie als elektromagnetische Strahlung ab. 




\subsection{Wellennatur der Materie}

\subsubsection{De Broglie-Wellenlänge}

Die Hypothese wurde aufgestellt, dass auch die Bewegung von Teilchen Wellencharakter besitzt, dass es auch bei bewegten Teilchen diesen Dualismus zu beobachten gibt. Es existiere also eine Materiewelle mit einer gewissen Gruppengeschwindigkeit. Die DE BROGLIE-Wellenlänge ist definiert als
\begin{equation}
\lambda = \frac{h}{mv} = \frac{h}{p}
\end{equation}
Der experimentelle Beweis wurde über \textbf{Interferenzerscheinungen} von Teilchenstrahlen erbracht, und zwar zunächst bei der \textbf{Beugung von Elektronenstrahlen} an Einkristallen. \\
Bei der BRAGGschen Reflexion entstehen Interferenzmaxima, die BRAGGsche Reflexionsbedingung lautet
\begin{equation}
k \cdot \lambda = 2 \ d \ \text{sin }\varphi
\end{equation}


\subsubsection{Wellenfunktionen und HEISENBERGsche Unschärferelation}
Diese besagt
\begin{center}
\textit{Es ist nicht möglich, Ort und Impuls eines dualistischen Teilchens gleichzeitig genau zu bestimmen.}
\end{center}
Die HEISENBERGsche Unschärferelation lautet
\begin{equation}
\Delta x \cdot \Delta p \geq h
\end{equation}

\subsection{Quantenmechanik}
\subsubsection{SCHRÖDINGER-Gleichung}

Es wurde aufgrund des Welle-Teilchen-Dualismus eine Wellenfunktion $\Psi_{(x,y,z)}$ eingeführt. Für einen bestimmten Ort schwingt die Amplitude \textbf{periodisch}. Die zeitfreie Amplitudengleichung lautet
\begin{equation}
\Delta \Psi = - \frac{\omega^2}{u^2}\Psi
\end{equation}
Die zeitfreie SCHRÖDINGER-Gleichung lautet
\begin{equation}
\Delta \Psi + \frac{2m}{\hbar}(E-U) \Psi = 0
\end{equation}
Sie gilt als Bestimmungsgleichung für die erlaubte Gesamtenergie E und die zugehörige Wellenfunktion $\Psi$. U ist die potentielle Energie des untersuchten Teilchens. Es stellt sich heraus, dass nicht beliebige Werte für die Gesamtenergie möglich sind, die Energiestufung ist allerdings nur bei mikroskopischen Objekten von Bedeutung. 
\\
Die Gleichung liefert neben der Energie auch zu jedem Eigenwert die zugehörige Eigenfunktion $\Psi$. Das Quadrat dieser Funktion liefert die \textbf{Aufenthaltswahrscheinlichkeit} eines Teilchens an einem bestimmten Ort. 

\subsubsection{Harmonischer Oszillator}
Die Wahrscheinlichkeit in der klassischen Physik das "Teilchen" innerhalb der Auslenkung zu finden ist 1. Wahrscheinlichkeit steigt zum Rand hin an, außerhalb ist sie Null. 
Für die quantenmechanische Behandlung des harmonischen Oszillators setzt man die potentielle Energie $U(x)$ in die SCHRÖDINGER-Gleichung ein
\begin{equation}
-\frac{\hbar^2}{2m}\frac{d^2\Psi(x)}{dx^2} + \frac{1}{2}kx^2\Psi_{(x)} = E\Psi_{(x)}
\end{equation}
Es gibt dabei gewisse Energieniveaus
\begin{equation}
E_n = (n + \frac{1}{2})h\nu
\end{equation}
Dabei ist zu beachten, dass nur \textbf{diskrete Energiestufen} erlaubt sind, der Abstand zwischen benachbarten \textbf{Energieniveaus} ist
\begin{equation}
\Delta E = h \cdot \nu
\end{equation}
Die Wellenfunktion ist entweder symmetrisch oder antisymmetrisch zum Ursprung und es gibt auch eine Aufenthaltswahrscheinlichkeit in klassisch verbotenen Gebieten. 

\pagebreak



\subsubsection{Tunneleffekt}

Dabei wird ein Teilchen betrachtet, dass auf eine Potentialbarriere zuläuft. Für $E < E_{Barriere}$ kann das Teilchen die Potentialbarriere nicht überwinden, es existiert also keine Aufenthaltswahrscheinlichkeit auf der anderen Seite der Barriere. \\
In der Quantenmechanik gibt es aber auch eine Wahrscheinlichkeit, dass das Teilchen auch bei kleinerer Energie die Barriere überwinden kann (es "durchtunnelt" die Barriere, diese Erscheinung nennt man \textbf{Tunneleffekt}) oder auch mit größerer Energie reflektiert wird. 
Dieser Effekt wird in der Elektronik bei der \textbf{Tunneldiode} ausgenutzt. 
 



\section{Atomphysik}
\subsection{Atommodelle}
\subsubsection*{THOMSON}
Atome sind zusammengesetzt aus positiven und negativen Ladungen, genannt das Rosinenkuchenmodell, die positiven Ladungen sind dabei gleichmäßig über das Volumen verteilt und die negativen Ladungen nehmen Gleichgewichtslagen an, um die sie stabile Schwingungen ausführen können. 

\subsubsection*{Dynamidenmodell von LENARD}
Aus Kathodenstrahlversuchen durch Materie (dünne Folien) schloss man, dass Atome viel leeren Raum haben, es wurde als Anhäufung positiv und negativ geladener "Kügelchen" betrachtet. 

\subsubsection*{RUTHERFORD}
RUTHERFORD führte Streuversuche durch mit den damals neu entdeckten $\alpha$-Teilchen an dünnen Folien und beobachtete selten eine Ablenkung um große Winkel. Die Folgerung daraus war, dass die positive Ladung an einer Stelle im Atom konzentriert sein musste $\Rightarrow$ Atomkern. Die Elektronen befänden sich auf Kreisbahnen um den Kern. \\
Die Problematik mit diesem Modell war, dass nach der klassischen Physik diese Elektronen aufgrund von Energieverlusten irgendwann in den Kern stürzen müssten.  

\subsubsection*{BOHR}
Dieser erweiterte das RUTHERFORDsche Atommodell um zwei Postulate, die die \textbf{Stabilität durch Quantelung} erklären.
\begin{center}
 \textit{Es gibt \textbf{bestimmte, stabile Bahnen}, auf denen das Elektron \textbf{strahlungsfrei} umlaufen kann.}
 \end{center} 
 
 \begin{center}
 \textit{Beim \textbf{Übergang} von einer Bahn höherer Energie zu einer Bahn geringerer Energie wird die \textbf{Differenz der Bahnenergien} als \textbf{Lichtquant} emittiert (Umkehr: Absorption)}. 
 \end{center}
 
 Beide Postulate lassen sich nicht mit der klassischen Physik vereinigen, die Kreisbewegung ist eine beschleunigte Bewegung und sollte zu einer Abstrahlung von elektromagnetischer Energie führen und diese müsste mit der Umlauffrequenz der Elektronen erfolgen. 
 
BOHR nimmt an, dass sich die Elektronen auf \textbf{Kreisbahnen um den Kern} bewegen. Für die \textbf{Wechselwirkung} zwischen Kern und Elektronen gelten die Gesetze der klassischen Mechanik und Elektrostatik.  
 

\subsection{Spektrum des H-Atoms}

\subsubsection*{BALMER und RYDBERG-Formel}

Atome strahlen diskrete, für das jeweilige chemische Element charakteristische Frequenzen ab, darauf begründet sich die \textbf{Chemische Spektralanalyse}. \\
Charakteristische Frequenzen bei der chemischen Spektralanalyse, die empirische Beschreibung der Abfolge der Wellenlängen gelang BALMER mit 
\begin{equation}
\lambda = 3646 \cdot \frac{n^2}{n^2 - 4}[\mathring{A}]
\end{equation}
n ist hierbei eine Laufvariable, die identisch ist mir der später eingeführten Hauptquantenzahl. \\
Die \textbf{Wellenzahl} ist definiert als die Anzahl der Wellenlängen pro Längeneinheit
\begin{equation}
\overline{\nu} = \frac{1}{\lambda} \text{ Wellenzahl = Anzahl der Wellenlängen pro cm}
\end{equation}
RYDBERG formte die BALMER-Formel um und erhielt
\begin{equation}
\overline{\nu} = \overline{R_H}\left( \frac{1}{2^2}- \frac{1}{n^2}\right)\text{ mit } \overline{R_H} \text{ der RYDBERG-Konstante}
\end{equation}



\subsubsection*{Termschema}
Separiert man die in der RYDBERG-Formel vorkommenden mathematischen Terme, ist die Aussendung von Lichtquanten der Energie $h \cdot \nu$ deutbar als Übergang zwischen zwei \textbf{Energietermen}. In einem sogenannte Termschema werden die Energieterme als waagrechte Linien eingetragen, die im Spektrum beobachteten Spektrallinien können als Übergänge zwischen diesen betrachtet werden. Als Energienullpunkt wird die \textbf{Ionisierungsenergie} verwendet, das ist jene Energie, die aufzuwenden ist, um das Elektron vom Grundzustand (n=1) in den Zustand n = $\infty$ (freies Elektron mit $E_{kin} = 0$) anzuheben. Diese wird in Elektronenvolt (eV) angegeben
\begin{equation}
E_{ion} = \frac{1}{(4 \pi \epsilon_0)^2}\frac{me^4}{2\hbar^2}Z^2
\end{equation}
Die Energie der Terme mit $n > 1$ lässt sich somit angeben mit
\begin{equation}
E_n = -E_{ion}\frac{1}{n^2}
\end{equation}

\pagebreak


\subsubsection*{FRANK-HERTZ-Versuch}
FRANCK und HERTZ gelang der erste, nicht-spektroskopische Beweis der Existenz diskreter Energiestufen(-niveaus) bei Atomen. In Quecksilberdampf driften die aus der geheizten Kathode C austretenden Elektronen unter dem Einfluss der zwischen C und der Anode A liegenden Spannung U und stoßen nach Durchlaufen der "mittleren freien Weglänge" mit Hg-Atomen zusammen. Die Anode ist als Gitter ausgeführt und die Elektronen können hindurchtreten und aufgrund ihrer kinetischen Energie gegen die Gegenspannung anlaufen, was einen Strom verursacht, der messbar ist. \\
Ist die \textbf{Energieaufnahme} der Elektronen \textbf{kleiner als die erste Anregungsenergie} der Quecksilberatome, so können nur \textbf{elastische Stöße} stattfinden. Es wird dabei keine Energie abgegeben. \\
Überschreitet die kinetische Energie der Elektronen einen Mindestwert, so können die Elektronen einen Teil ihrer Energie in Form von Anregungsenergie übertragen, es finden daher \textbf{inelastische Stöße} statt, sie kommen mit verringerter Energie zur Anode und der \textbf{Strom sinkt}. Bei Quecksilber liegen die Maxima 4.9V auseinander, die Anregungsenergiestufe liegt somit bei 4.9 eV. Beim Zurückfallen wird Strahlung abgegeben, es gilt dabei
\begin{equation}
\Delta E = h \cdot \nu = \frac{h \cdot c}{\lambda} = 4.9 eV \quad \Rightarrow \quad \lambda=2520 \mathring{A} \text{ zu erwarten}
\end{equation}



\subsection{Lösung der Schrödinger Gleichung für ein Elektron}
Verwendet wird hierbei die zeitfreie SCHRÖDINGER-Gleichung
\begin{equation}
\Delta \Psi + \frac{2m}{\hbar^2}(E-U)\Psi = 0
\end{equation}
Da das Elektron im COULOMG-Feld des Kern ist, muss die Schrödinger Gleichung in Kugelkoordinaten umgerechnet werden. Jetzt müssen physikalische Randbedingungen eingeführt werden
\begin{itemize}
\item Eindeutigkeit auf der Kugeloberfläche
\item Verschwinden der Funktion im Unendlichen
\item Normierung der Aufenthaltswahrscheinlichkeit
\end{itemize}
Die \textbf{Wellenfunktion} muss in ein \textbf{Produkt von Funktionen mit je einem Parameter} zerlegt werden, eine \textbf{Separationskonstante} $A = \ell(\ell+1)$ wird errechnet. $\ell$ ist die \textbf{Bahndrehimpuls-Quantenzahl}. Eine weitere Konstante wird eingeführt: $B = m_\ell^2$ wobei $m_\ell$ die \textbf{magnetische Bahndrehimpuls-Quantenzahl} ist. \\
Nun kann man die komplexe Schrödingergleichung in Kugelform in drei einfachen Differentialgleichungen anschreiben. Nun werden die Randbedingungen umgesetzt. \\
$m_\ell$ muss eine \textbf{ganze Zahl} sein, daraus kann man die Existenz ganzer Quantenzahlen ableiten. Als Einschränkung muss gelten
\begin{equation}
- \ell \leq m_\ell \leq \ell
\end{equation}

Im Radiusteil der Gleichung ist die \textbf{Gesamtenergie} $E = E_{kin} + U$ enthalten. 
\begin{itemize}
\item $E  >  0$: Das Elektron ist nicht an den Kern gebunden und besitzt kinetische Energie
\item $E  >  0$: Das Elektron ist an den Kern gebunden, die Gleichung ist aber nur für bestimmte Werte lösbar (Quantelung)
\item $E  =  0$: Elektron unendlich weit vom Kern entfernt
\end{itemize}
Unter der Erfüllung der Bedingungen für sinnvole Lösungen ergeben sich die Energieeigenwerte zu
\begin{equation}
E_n = -h \cdot c \cdot \overline{R}_\infty \cdot \frac{Z^2}{n^2}
\end{equation}
n ist hierbei die Hauptquantenzahl und für $\ell$ gilt
\begin{equation}
0 \leq \ell \leq n-1
\end{equation}
Die Schrödinger-Gleichung liefert also
\begin{itemize}
\item Energiequantelung
\item Quantenzahlen $n,\ell,m_\ell$
\item Das Ergebnis der Energieeigenwerte ist das selbe wie bei den BOHRschen Berechnungen
\end{itemize}
$\ell$ kann deshalb als Drehimpuls angesehen werden, da nach Aufstellen einer Operatorgleichung für $|\vec{L}_\ell|$ $\ell$ mit dem Winkelteil der separierten Energiegleichung übereinstimmt. 


\subsection{Ladungsdichteverteilung des Elektrons um den Atomkern}

Das Quadrat der Wellenfunktion ist die räumliche Aufenthaltswahrscheinlichkeit für ein Elektron, man spricht von einer "Ladungswolke" nach der SCHRÖDINGERschen Vorstellung, diese ist nicht eben sondern hat eine räumliche Verteilung.  

\subsubsection*{Azimutale Verteilung}
Der Winkelanteil der Schrödinger-Lösung gibt den Winkel an, den die Ladung bevorzugt. Ein \textbf{Orbital} ist die räumliche Gestalt der Elektronenwolke.

\subsubsection*{Radiale Ladungsdichteverteilung}
Dies gibt die Wahrscheinlichkeit für den Aufenthalt des Elektrons innerhalb einer Kugelschale an. Es zeigt sich eine Übereinstimmung mit den BOHRschen Radien: Die maximale Aufenthaltswahrscheinlichkeit liegt für $\ell = n-1$ gerade bei den BOHRschen Radien(genau ein Maximum). Bei $\ell = 0$ gibt es $n$ Maxima. Bei $n=4,\ell = 0$ (4s-Zustand) liegt das Hauptmaximum weit außerhalb des Zustandes $n=3, \ell = 2$, besitzt aber Nebenmaxima in Kernnähe. Das 4s-Eletron ist bei schwereren Kernen daher stärker gebunden als ein 3d- Elektron. 
\pagebreak

\subsection{Erweiterung bisheriger Betrachtungen}
Zur Berechnung der Wasserstoff-Feinstrukturterme muss man relativistische Eigenschaften betrachten
\begin{itemize}
\item \textbf{Eigendrehimpuls} (Spin)
\item \textbf{magnetisches Moment} $\mu_s$ des Elektrons
\end{itemize}
Zur Berechnung der gesamten Elektronenenergie ist also nicht nur die Bahnenergie (kinetische Energie) und die potentielle Energie maßgebend, sondern auch noch eine Zusatzenergie, die von der \textbf{Wechselwirkung zwischen Bahndrehimpuls und Eigendrehimpuls} herrührt. 
\begin{center}
\textit{Unter Berücksichtigung des Elektronenspins s = $1/2$ sind \textbf{$4$ Quantenzahlen} notwendig, um einen Elektronenzustand zu beschreiben. }
\end{center}


\subsection{Bedeutung der Quantenzahlen}

\begin{itemize}
\item \textbf{Hauptquantenzahl n}: Die Energie ist näherungsweise proportional zu $\frac{1}{n^2}$
\item \textbf{Bahndrehimpulsquantenzahl $\ell$}: Das Elektron besitzt auf seiner \textbf{Bahnbewegung} einen \textbf{Bahndrehimpuls} $\vec{L}_\ell$. Für den Betrag gilt $|\vec{L_\ell}| = \sqrt{\ell (\ell + 1)}\cdot \hbar$ mit $\ell = 0,1,2,...,n-1$. Allgemein für quantenmechanische Drehimpulse gilt \\ $|\vec{L_q}| = \sqrt{q (q + 1)}\cdot \hbar$
\item \textbf{Spinquantenzahl s}: $|\vec{L_s}| = \sqrt{s (s + 1)}\cdot \hbar$, kann nur den Wert s = 1/2 annehmen
\item \textbf{Orientierungsquantenzahl für Bahndrehimpuls $m_l$}: Projektion des Winkels: $|\vec{L_{\ell,z}}| = m_l \cdot \hbar$
\item \textbf{Orientierungsquantenzahl für Elektronenspin $m_s$}: $|\vec{L_{\ell,s}}| = m_s \cdot \hbar$ mit $m_s = \pm 1/2$
\end{itemize}

\subsection{Spektrallinien, Auswahlregeln für Dipolstrahlung}
Die Energiedifferenz zwischen zwei Elektronenbahnen ist definiert als
\begin{equation}
E_n - E_{n'} = h \cdot \nu
\end{equation}
Die abgestrahlten Frequenzen entsprechen den Spektrallinien, die Wellenlänge gibt aber nur Information über die Energiedifferenz ab. Es gibt \textbf{Auswahlregeln} für die \textbf{optischen Übergänge} zwischen atomaren Niveaus. Klassisch würde ein zeitlich veränderlicher elektrischer Dipol (Elektronen auf Kreisbewegung um den Kern) ständig elektromagnetische Energie abstrahlen, quantenmechanisch gibt es aber stabile Bahnen ohne Strahlung, zu Strahlung kommt es nur bei einem Übergang zwischen Bahnen. Man kann aus der \textbf{Wellenfunktion des Anfangs- und Endzustandes} die \textbf{Übergangswahrscheinlichkeit} berechnen. 
Die Auswahlregeln für das Einelektronensystem sind somit
\begin{itemize}
\item $\Delta \ell = \pm 1$
\item $\Delta j = 0,\pm 1$
\item $\Delta s = 0$
\end{itemize}


\pagebreak

\subsection{Chemische Elemente mit einem Elektron außerhalb einer Schale}
Das Elektron auf einer Kreisbahn um den Kern repräsentiert eine Strom I, da die Stromschleife sehr klein ist, entspricht sie daher einem rotierenden magnetischen Dipol. Das \textbf{magnetische Moment} eines Elektrons ist mit seinem \textbf{mechanischen Moment} verknüpft. Das \textbf{magnetische Moment der Bahn} erhält man mit
\begin{equation}
\vec{\mu_\ell} = -\frac{\mu_B}{\hbar} \cdot \vec{L_\ell} \text{  mit  } \mu_B = \frac{\mu_0\cdot e}{2m} \cdot \hbar \text{ das BOHRsche Magneton}
\end{equation}
Das magnetische Moment steht immer \textbf{antiparallel zum mechanischen Drehimpuls}. \\
\subsubsection*{Termschemen in Alkalimetallen}
Sind Atome mit einem Elektron mehr als Edelgase, welche eine \textbf{abgeschlossene Elektronenschale} besitzen und dadurch Gesamtdrehimpuls und Gesamtspin gleich Null haben. Bei Alkalimetallen ist das nun gleich, nur mit einem zusätzlichen "Leuchtelektron". Daher gibt es ähnliche Spektren wie bei Wasserstoff. \\
Sind $n$ und $\ell$ des Leuchtelektrons groß,so bewegt sich das Leuchtelektron auf kreisähnlichen Bahnen weit außerhalb des Elektronenrumpfes,die Energie des Elektrons $\approx$ Wasserstoff. \\
Sind $n$ und $\ell$ klein, "taucht" das Leuchtelektron in den Elektronenrumpf ein, auf einem Teil der Bahn gibt es eine \textbf{Kernladungszahl} $Z_{eff} > 1$. Die Folgen sind
\begin{itemize}
\item Energieabsenkung der Niveaus relativ zur Energie der Wasserstoff-Terme gleicher Hauptquantenzahl
\item Die Energie ist stark vom Wert von $\ell$ abhängig
\item Die Dublettaufspaltung ist viel größer als bei Wasserstoff aufgrund der hohen Spin-Bahn-Wechselwirkung
\end{itemize}
Quantenmechanisch gilt: 3s-Elektron hat Maxima der Aufenthaltswahrscheinlichkeit in Kernnähe, 3d eine kleinere Aufenthaltswahrscheinlichkeit. 

\subsection{Zweielektronensysteme}

\subsubsection*{Quantenmechanische Betrachtung des Heliumatoms}
Zunächst ist die potentielle Energie der Elektronen zu ermitteln, diese wird dann in die SCHRÖDINGER-Gleichung eingesetzt. Eine exakte Lösung ist unmöglich, daher eine Näherung. Bei der 0. Näherung werden die Elektronen \textbf{unabhängig betrachtet}. Dabei wird die berechnete Bindungsenergie zu groß. Bei der 1. Näherung werden die Elektronen noch immer als unabhängig betrachtet, aber die Kernladung wird durch jeweils ein anderes Elektron teilweise abgeschirmt. Die Elektronen können nicht unterschieden werden, daher muss die Aufenthaltswahrscheinlichkeit gleich groß und symmetrisch sein.

\subsubsection*{PAULI-Verbot}
\begin{center}\textit{In einem Atom können 2 Elektronen nicht den selben Satz von Quantenzahlen haben, sie müssen sich in mindestens einer Quantenzahl voneinander unterscheiden.}\end{center} 


\subsection{Periodensystem}
\subsubsection*{BOHRsches Aufbauprinzip}
Beim Übergang von einem Element zum nächsten im Periodensystem wird jeweils ein Elektron und ein Proton hinzugefügt, dabei muss das PAULI-Prinzip berücksichtigt werden (wenigstens eine der 4 Quantenzahlen muss sich von den bereits "verbrauchten" Quantenzahlen unterscheiden). Elektronen mit gleicher Hauptquantenzahl fasst man zu einer "Schale" zusammen. 
Die Anzahl der Elektronen pro Schale ergibt sich zu
\begin{equation}
z = 2 \cdot n^2
\end{equation}
Nach der \textbf{HUNDschen Regel} werden immer freie Plätze in Schalen mit geringster Hauptquantenzahl zuerst gefüllt. Die 4s-Schale wird vor der 3d Schale gefüllt, da die Aufenthaltswahrscheinlichkeit der 4s-Elektronen in Kernnähe ein Maxima hat. 

\subsection{Atom in äußeren Feldern}
\subsubsection*{STERN-GERLACH-Versuch}
Ein Atomstrahl wird durch ein Magnetfeld in zwei Teilstrahlen aufgespalten, diese Spaltung ist Beweis für die Existenz \textbf{halbzahliger Drehimpulse} (Spin s = 1/2), es gibt also nur zwei mögliche Orientierungen ($\pm 1/2$). 

\subsubsection*{Atome in Magnetfeldern (ZEEMAN-Effekt)}
In einem Magnetfeld $\vec{B}$ besitzt ein magnetischer Dipol die Energie
\begin{equation}
E = \vec{\mu} \cdot \vec{B}
\end{equation}
Durch das Feld wird eine \textbf{Raumrichtung physikalisch ausgezeichnet} und es erfolgt eine \textbf{Richtungsquantelung}.
\begin{equation}
|\vec{L}_{J,z}| = m_J \cdot \hbar
\end{equation}
Es existiert daher ein in z-Richtung wirksames magnetisches Moment des Atoms. Ein Energieniveau mit Gesamtdrehimpuls $\vec{L_J}$ spaltet daher in einem äußeren Magnetfeld in $2J+1$ Subniveaus unterschiedlicher Energie auf. Wegen der Spaltung der Feinstrukturniveaus \textbf{spalten }auch die beobachteten \textbf{Spektrallinien} in \textbf{mehrere Komponenten} auf. \\
Beim Übergang zwischen Singulett-Termen (S = 0, r = 1) tritt der \textbf{normale ZEEMAN-Effekt} auf: Die Energieaufspaltung $\Delta E$ für das untere und obere Niveau ist gleich groß, verschiedene Übergänge haben die selbe Übergangsfrequenz und fallen zusammen. 

Im Linienspektrum sind drei Linienkomponenten zu erkennen, abgestrahltes Licht ist bei gleichbleibender magnetischer Quantenzahl horizontal, sonst senkrecht, zur Feldrichtung polarisiert. 

\subsubsection*{Atome in elektrischen Feldern (STARK-Effekt)}
Atome, die in gewissen Elektronenzuständen ein \textbf{permanentes, elektrisches Dipolmoment} besitzen, verhalten sich analog zu Atomen in magnetischen Feldern: die Zusatzenergie im Feld ist proportional zur Feldstärke (linearer STARK-Effekt mit $\Delta E \propto E$). Man beobachtet eine \textbf{symmetrische Energieaufspaltung} der Niveaus ähnlich dem ZEEMAN-Effekt. \\
Ist kein permanentes Dipolmoment vorhanden, wird eines induziert. Die Spektrallinien werden aufgespalten und verschoben(quadratischer STARK-Effekt mit $\Delta E \propto E^2$). 

\subsection{Strahlende Übergänge}
\subsubsection*{EINSTEINs Betrachtung}
\begin{itemize}
\item \textbf{Absorption}: Das Atom absorbiert ein Quant der elektromagnetischen Strahlung und erhöht seine innere Energie um $\Delta E = h \cdot \nu$
\item \textbf{Stimulierte Emmision}: Hier wird ein Atom im angeregten Zustand durch ein Strahlungsfeld stimuliert und gibt seine Anregungsenergie in Form von Strahlung ab. Das emittierte Photon ist mit dem stimulierenden Photon identisch (kohärente Abstrahlung). 
\item \textbf{Spontane Emmision}: Hier kehren angeregte Atome nach gewisser Zeit unter Emission eines Lichtquants in den Grundzustand zurück, der Zeitpunkt ist nicht bestimmbar.
\end{itemize}

\subsection{Röntgenspektren}
Erzeugung durch Verzögerung von schnellen Elektronen im Feld der Atome eines Festkörpers. Umwandlung der kinetischen Energie in Lichtquanten, die Energie der Photonen ist über einen großen Bereich verteilt, dadurch erhält man kontinuierliche RÖNTGENspektren. \\
Zuerst wird das aus der Kathode austretende Elektron im zwischen Antikathode und Kathode anliegenden elektrischen Feld beschleunigt
\begin{equation}
e \cdot U_0 = \frac{m_e \cdot v_1^2}{2}
\end{equation}
Danach wird ein Teil der kinetischen Energie in Photonenenergie umgesetzt
\begin{equation}
E_{Photon} = h \cdot \nu = \frac{m_e \cdot v_1^2}{2}- \frac{m_e \cdot v_2^2}{2}
\end{equation}
Die Maximalenergie der Photonen und damit die Grenzfrequenz des RÖNTGEN-Bremskontinuums ergibt sich für $v_2 = 0$
\begin{equation}
h \cdot \nu_G = \frac{m_e \cdot v_1^2}{2} = e \cdot U_0
\end{equation}
Die \textbf{Intensität} der RÖNTGENstrahlung ist von der Anzahl der pro Zeiteinheit auf die Antikathode treffenden Elektronen und damit vom Heizstrom der Kathode abhängig.
\\
Aus einer Bestimmung der kürzesten Wellenlänge $\lambda_G$ kann sehr genau das \textbf{PLANCKsche Wirkungsquantum h} bestimmt werden
\begin{equation}
h = \frac{e \cdot U_0 \cdot \lambda_G}{c}
\end{equation}

\subsubsection*{Entstehung des charakteristischen Spektrums}
Überschreitet die Beschleunigungsspannung einen bestimmten, vom Material der Antikathode abhängigen Wert, so treten im Röntgenspektrum neben dem Bremskontinuum zusätzliche, diesem überlagerte \textbf{RÖNTGENlinien} diskreter Wellenlänge auf, die sogenannte \textbf{charakteristische RÖNTGENstrahlung}. Diese RÖNTENlinien entstehen durch Stoß mit Elektronen der Antikathoden-Atome und das Herausschlagen derer Elektronen. Beim Ersetzen des Lochs mit Elektronen höherer Schalen wird Strahlung abgegeben, ähnlich dem Emissionsspektrum von Wasserstoff. 

\subsubsection*{Spektrale Zerlegung der RÖNTGENstrahlung}
Die Absolutbestimmung von Wellenlängen erfolgt im optischen Bereich am genauesten durch Beugung am Gitter. Linsen sind nicht einsetzbar, da Röntgenstrahlen das Material ungehindert durchdringen. Daher Beugung bei Kristallen mit Atomabstand von ca $1 \mathring{A}$. Grundlage für diese Messungen ist die \textbf{BRAGGsche Reflexionsbedingung}
\begin{equation}
k \cdot \lambda = 2 \ d \sin \varphi
\end{equation}
Atome des Kristallgitters können hierbei als Sekundärstrahler genutzt werden. Bei bestimmten Strahlungsrichtungen erhält man konstruktive Interferenz. 
Alternativ ist auch die Beugung an einem optischen Gitter (ca 600 Linien/mm) möglich, wenn dieses sehr flach bestrahlt wird. 

\subsubsection*{Absorption von Röntgenstrahlung}
Beim Durchgang durch Materie findet eine Schwächung statt, es gilt das exponentielle Schwächungsgesetz
\begin{equation}
I_d = I_0 \cdot e^{-\kappa \cdot d} \text{ mit Schwächungskoeffizient } \kappa \propto Z^4 \cdot \lambda^3
\end{equation}
Daher eignet sich zur \textbf{Abschirmung} ein Material mit \textbf{hoher Kernladungszahl} wie zum Beispiel Blei. \\
Das Absorptionsvermögen eines Elementes nimmt sprunghaft zu, wenn die Energie der durchdringenden Strahlung einen gewissen Wert erreicht hat, zur Absorption ist eine \textbf{Mindestenergie des Photons} notwendig. Je größer danach die kinetische Restenergie des abgetrennten Elektrons ist, desto unwahrscheinlicher wird die Absorption, es bildet sich eine \textbf{Absorptionskante}. 
Es gibt zwei Arten, einerseits die \textbf{echt Absorption}, wo Photonenenergie auf das Material übertragen wird (meist Umwandlung in Wärme) und \textbf{Streuabsorption}, wo eine Richtungsänderung des Photons stattfindet (COMPTON-Wellenlängenänderung). 

\subsubsection*{RÖNTGEN-Strukturanalyse}
LAUE setzte Kristallgitter ein und konnte mit Hilfe von RÖNTGENstrahlen bzw. der BRAGGschen Beugung die Atomanordnung von Kristallgittern erforschen. \\
Bei der \textbf{Drehkristallmethode} dreht man einen Kristall vor einer RÖNTGENstrahlung und misst die Maxima der Strahlung in Abhängigkeit des Einfallswinkels auf der anderen Seite, so kann auch die Gitterkonstante bestimmt werden. \\
beim \textbf{LAUE-Verfahren} wird ein Kristall durchstrahlt, es kommt zur Beugung am dreidimensionalen Punktgitter. Die Beugungsbilder zeigen Symmetrie des Kristalls, sind aber schwer auswertbar. 























\section{Laser}
\subsubsection*{Aufbau}
Das Kunstwort \textbf{Laser} leitet sich her von "\textbf{L}ight \textbf{A}mplification by \textbf{S}timulated \textbf{E}mission of \textbf{R}adiation". Ein Laser besteht aus drei wichtigen Teilen, einem \textbf{aktiven Medium}, das durch eine \textbf{Energiepumpe} in den aktiven Zustand versetzt wird, und einem \textbf{Resonator}, der das Strahlungsfeld des Lasers formt.
\subsection{Funktionsweise}
Hier wird nochmals die Wechselwirkung zwischen Licht und einem Atom, das zwei Energieniveaus $E_2$ und $E_1$($E_2 > E_1$) besitzen soll. Ein Atom kann mit einem Photon auf drei unterschiedliche Arten wechselwirken, wenn die Photonenenergie $h \cdot \nu$ gleich der Energiedifferenz $E_2 - E_1$ ist.
\begin{itemize}
\item Das Atom befindet sich im Grundzustand mit $E_1$ \\ 
\begin{itemize}
\item $W_{12} = \rho(\nu)\cdot B_{12}$   Wahrscheinlichkeit für \textbf{Absorption}
\end{itemize}
\item Das Atom befindet sich im angeregten Zustand mit $E_2$ \\
\begin{itemize}
\item $W_{21} = \rho(\nu)\cdot B_{21}$   Wahrscheinlichkeit für \textbf{induzierte Emission}
\item $W_{21,sp} = A_{21}$   Wahrscheinlichkeit für \textbf{spontane Emission}
\end{itemize}
\end{itemize}
Im Strahlungsfeld müssen Emissions- und Absorptionsvorgänge gleich wahrscheinlich sein. Die Anzahl der Vorgänge erhält man durch Multiplikation mit den Besetzungsdichten. Damit mehr Emissionsvorgänge als Absorptionsvorgänge ablaufen, muss also die Besetzungsdichte im angeregten Zustand größer sein als im Grundzustand, es muss \textbf{Besetzungsinversion} vorliegen.

\begin{center}
\textit{Ein Medium, in dem \textbf{Besetzungsinversion} vorliegt (\textbf{aktives Medium}), kann eine durchlaufende  Lichtwelle geeigneter Frequenz verstärken.}
\end{center}

Die \textbf{Energiepumpe} muss also dem oberen Zustand Laserübergangs ($E_2$) selektiv mehr Energie zuführen als dem Grundzustand des Übergangs. Man erhält eine Lichtwelle, wenn Photonen aus spontaner Emission mit \textbf{Hilfe von Spiegeln} wieder ins \textbf{aktive Medium zurückgeleitet} werden. Die rückkommenden Photonen bewirken \textbf{induzierte Emissionsvorgänge} und die Photonenanzahl wächst exponentiell, bis die Anzahl der neu erzeugten Photonen gleich der durch verschiedene Vorgänge verlorengegangenen Photonenzahl ist. \\
Die \textbf{spontane Emission} ist also notwendig für das "\textbf{Starten}" der \textbf{induzierten Absorption}. Für den Betrieb ist es aber notwendig, dass die meisten Atome im oberen Zustand verharren und nicht spontan in den Grundzustand übergehen. 
\begin{center}
\textit{Die \textbf{induzierte Emission muss wahrscheinlicher} sein als die \textbf{spontane Emission}}. 
\end{center}
Der Austritt aus dem Spiegelsystem ist möglich durch eine Öffnung oder halb durchlässigen Spiegel. \\
Um die \textbf{geforderte Energiedichte} zu erhalten, muss eine stehende Welle gebildet werden. Um eine stehende Welle zu bilden, muss das Strahlungsfeld eine bestimme Wellenlänge haben 
\begin{equation}
L = k \cdot \frac{\lambda}{2} \text{ mit k als ganze Zahl}
\end{equation}
Die \textbf{Anzahl der möglichen Schwingungszustände} in einem Frequenzintervall wird als \textbf{Moden} bezeichnet. Damit induzierte Emission wahrscheinlicher ist als spontane Emission, muss \textbf{die Photonenanzahl pro Mode größer als 1 werden}. \\
Dies kann entweder durch Erhöhung der Temperatur oder durch Einschränkung der Moden mit Resonatoren erreicht werden. 


\subsection{Offene Resonatoren}

Diese bestehen aus zwei gegenüberliegenden, im einfachsten Fall planen, Spiegeln mit i.a. sehr hohem Reflexionsvermögen. Die Normalvektoren der Spiegelebenen müssen exakt antiparallel zueinander ausgerichtet sein. 
\subsubsection*{Longitudinale Modenstruktur}
Die Phasendifferenz zwischen zwei Strahlen ist von der Länge abhängig. Die maximale Verstärkung ergibt sich bei stehenden Wellen mit 
\begin{equation}
k \cdot \lambda = 2 \cdot n \cdot L \text{ mit L Abstand der Spiegel, n Brechzahl des Mediums im Resonator}
\label{Resonanzbedingung}
\end{equation}
Immer dann, wenn der Resonator ein Transmissionsmaximum besitzt, kann sich im Inneren eine stehende Welle aufbauen. Der Resonator speichert in diesem Fall elektromagnetische Energie. 
Diejenigen Schwingungszustände, für die die Resonanzbedingung (6.2 G \ref{Resonanzbedingung}) gilt, nennt man \textbf{longitudinale Moden}. 

\subsubsection*{Transversale Modenstruktur}

Man betrachtet dabei die Beugung an den Spiegeln, der Resonator wird als Abfolge von Blenden betrachtet. Die Feldverteilung ist dabei gleich wie bei transversalen elektromagnetischen Moden (GAUSSsche Intensitätsverteilung). 

\pagebreak


\subsection{Schwellwerbedingungen, Lasermoden}

\subsubsection*{Schwellwertbedingung}
Man betrachtet einen ein Medium durchlaufenden Lichtstrahl. Normalerweise gilt das BEERsche Absorptionsgesetz
\begin{equation}
I(z) = I(z = 0)\cdot e^{-\alpha \cdot z}
\end{equation}
Damit es zu einer Lasertätigkeit kommt, muss die \textbf{Verstärkung} größer sein als die \textbf{Absorptionsverluste}. Die Intensität im Resonator muss daher einen Mindestwert (Laserschwelle) übersteigen. 

\subsubsection*{Lasermoden}
Die Bedingungen für die Emission von Laserlicht lauten
\begin{itemize}
\item Vorliegen von \textbf{Besetzungsinversion}, dies wird vom Medium für einen bestimmten Wellenbereich erfüllt (\textbf{Verstärkungsprofil} des aktiven Mediums)
\item \textbf{Verstärkung} ist nur für \textbf{Eigenfrequenzen des Resonators} möglich, also in den Frequenzbereichen der \textbf{longitudinalen Resonatormoden}
\item Das \textbf{Verstärkungsprofil} des aktiven Mediums muss \textbf{über der Laserschwelle} liegen 
\end{itemize}

\subsubsection*{Laser-Ausgangsleistung}
Betrachtet wird ein Vierniveau-System, die Energiepumpe hebt die Atome des Lasermediums in den maximalen zustand, von dort fallen sie (strahlungslos) zurück in den oberen Zustand des Laserübergangs. Von dort aus ist die induzierte und spontane Emission zum unteren Zustand möglich, ein Teil der Besetzungsdichte kann aber auch verloren gehen, diese Verluste werden durch die \textbf{Relaxionsrate} erfasst. Abschließend fallen sie wieder auf das Grundniveau zurück. 


\subsection{Lasertypen}
\subsubsection*{Helium-Neon-Laser}
Der Helium-Neon-Laser ist der am häufigsten eingesetzte Laser im sichtbaren Spektralbereich, das aktive Medium ist eine Niederdruck-Gasentladung, die in einer Mischung aus Helium und Neon betrieben wird. Die oberen Laserlinien-Niveaus werden durch Stöße mit metastabilen Helium-Atomen selektiv besetzt, wodurch im Neon-Atom zwischen Niveaugruppen 3s bzw. 2s und tieferen Niveaugruppen 3p/2p eine \textbf{Besetzungsinversion} entsteht. \\
Die \textbf{Helium-Niveaus} werden in der Gasentladung durch Elektronenstoß besetzt. Die selektive Energieübertragung erfolgt deshalb, weil die Anregungsenergien der genannten Neon-Niveau-Gruppen mit der Anregungsenergie der metastabilen Helium-Niveaus übereinstimmt. \\
Es kommt zu Emission von monochromatischen Licht einer Neon-Spektrallinie, aber mit Moden-Struktur(als eng benachbarte Frequenzen) (\textbf{Multimode-Betrieb}), trotzdem gibt es eine gute Kohärenzlänge. Mit zusätzlichen optischen Objekten lässt sich auch \textbf{Single-Mode-Betrieb} erreichen. 

\subsubsection*{Argon-Ionen-Laser}
Hoch angeregte Elektronen von ein-/zweifach geladenen Argon-Ionen werden durch Elektronenstoß stärker besetzt als darunterliegende Zustände. Vorteil ist die Möglichkeit der Erzeugung von \textbf{Dauerstrich-Laserlicht hoher Ausgangsleistung}, aber der Wirkungsgrad ist relativ gering. 

\subsubsection*{$CO_2$-Laser}
Ist einer der wichtigsten Lasertypen zur Materialbearbeitung, es sind \textbf{Dauerstrich-Leistungen von bis zu 100kW} möglich, ein weiterer Vorteil ist der relativ gute Wirkungsgrad. Die Anregung der $CO_2$-Moleküle erfolgt wieder in einer Gasentladung, die in einem Gemisch aus $CO_2$, $N_2$ und He brennt. Überbesetzung entsteht hauptsächlich durch Stöße aus $CO_2$ und $N_2$. Hier werden aber keine elektronischen Übergänge besetzt, sondern die Laserübergänge erfolgen zwischen Schwingungs-Rotations-Niveaus. 


\subsubsection*{Titan-Saphir-Laser (Pulsbetrieb)}

Es ist möglich, Laserpulse mit sehr kurzer Dauer (ca. 1fs) auszusenden. Für Untersuchungen von \textbf{Dynamik schnel ablaufender Vorgänge} eignet sich dieser Laserart also besonders. 

\subsubsection{Halbleiterlaser}
Laserdioden werden mittles Halbleitertechnologie erzeugt, die wichtigsten Eigenschaften sind
\begin{itemize}
\item Massenproduktion möglich
\item Kleine Abmessungen und damit leichte Integrierbarkeit möglich
\item Direkte Anregung mit kleinen Spannungen und Strömen
\item Hoher Wirkungsgrad
\item Direkte Modulationsmöglichkeit der emittierten Intensität\\ Anwendung zur \textbf{Datenübertragung}, Modulation bis 10 GHz
\end{itemize}
Die Elektronen halten sich in Energiebändern auf, bei p-Halbleitern ist das untere Band (\textbf{Valenzband}) mit positiven Löchern dotiert, bei n-Halbleitern dagegen das \textbf{Leitungsband} mit beweglichen Elektronen. Zwischen den Bändern besteht ein verbotener Bereich. Bringt man p- und n-Halbleiter in Kontakt, so diffundieren Elektronen so lange ins p-Gebiet, bis die FERMI-Kanten zusammenfallen und eine Raumladung und damit eine \textbf{Potentialdifferenz} entsteht. 
Bei der \textbf{Rekombination} kann die Energiedifferenz in Form von Photonen ($E = h \cdot \nu$) abgegeben werden. Bei Spannung in Durchlassrichtung entsteht eine Zone mit \textbf{Besetzungsinversion}, dadurch wird eine Lasertätigkeit möglich. Dazu muss aber ein gewisser Schwellstrom überschritten werden, bei kleinen Strömen gibt es viele Lasermoden, bei zunehmenden Strom setzt sich ein einzelner Lasermode durch. 





















\pagebreak

\section{Kern-Elementarteilchenphysik}
\subsection{Grundsätzliches über Atomkerne}
Ein Atomkern wird durch seine \textbf{Kernladungszahl}(=\textbf{Ordnungszahl}) Z (=\textbf{Anzahl der Protonen}) und seine \textbf{Massenzahl} MZ (=\textbf{Anzahl der Protonen und Neutronen}) charakterisiert. Die Schreibweise lautet beispielsweise $_2He^4$. Die \textbf{Kernladungszahl} ist auch aus Streuexperimenten mit $\alpha$-Teilchen an Kernen bestimmbar. Die Massenzahl gibt die gerundete relative Masse an. 
\begin{itemize}
\item \textbf{Isotope}: Kerne gleicher Ordnungszahl, aber verschiedener Masse
\item \textbf{Isobare}: Kerne gleicher Masse, aber verschiedener Ordnungszahl
\item \textbf{Isomere}: Kerne in verschiedenen Anregungszuständen
\end{itemize}

\subsubsection*{Aufbau der Kerne}
Kerne sind aus Protonen und Neutronen zusammengesetzt, Umwandlung von Neutronen in Protonen und umgekehrt ist möglich, daher wurde der Überbegriff \textbf{Nukleon} eingeführt. Der Zustand eines Nukleons wird spezifiziert durch die \textbf{Isospinquantenzahl $T_Z$}:
\begin{itemize}
\item $T_Z$ = 1/2  kennzeichnet Protonen
\item $T_Z$ = -1/2   kennzeichnet Neutronen
\end{itemize}

\subsubsection*{Radius, Dichte und Form}
Der Radius des Kerns wird definiert als entweder der Abfall der Dichte oder der Reichweite der Kernkräfte auf die Hälfte. Messbar ist dieser durch Messung der Winkelverteilung von gestreuten $\alpha$-Teilchen oder Abstoßung dieser unter Annahme reiner COULOMB-Abstoßung 
\begin{equation}
E_{kin} = E_{pot} = \frac{(Z-2)e\cdot2e}{4 \pi \epsilon_0 \cdot r_0} \ \Rightarrow \ r_0
\end{equation}
Die Dichte ist weitgehend für alle Kerne konstant mit $\approx 2 \cdot 10^{23}kg/m^3$ und die Form ist zunächst als \textbf{kugelförmig} anzunehmen, aber nur grob, eher in Form von Rotationsellipsoiden.

\subsubsection*{Kerndrehimpuls}
Der Kern besitzt einen \textbf{mechanischen Drehimpuls }, der durch die Angabe der \textbf{Kernspinquantenzahl I} festgelegt ist
\begin{equation}
|\vec{L_I}| = \sqrt{I(I+1)}\cdot \hbar
\end{equation}

Auch angeregte Zustände von Kernen sind möglich, Kerne mit gleicher Ladung und Masse, aber in verschiedenen Anregungszuständen (verschiedener innerer Energie) und daher mit verschiedener Stabilität nennt man \textbf{isomere Kerne}. 

\subsubsection*{Magnetische Momente}

Mit dem mechanischen Drehimpuls des Kerns ist auch ein \textbf{magnetisches Moment} verbunden, es Einheit definiert man das sogenannte \textbf{Kernmagneton}. Trotz \textbf{fehlender elektrischer Ladung} besitzt das Neutron ein \textbf{magnetisches Moment}. 

\subsubsection*{Massendefekt und Kernbindungsenergie} 
Erwartungsgemäß sollte ja gelten $M_K = M_P + M_N$, diese ist aber kleiner, die Differenzmasse bezeichnet man als Massendefekt. Daraus ergibt sich die Bindungsenergie über 
\begin{equation}
E = \Delta m \cdot c^2
\end{equation}
Helium-Kerne besitzen eine große Bindungsenergie (28eV) und sind daher sehr stabil, Deuterium-Kerne hingegen besitzen eine geringe Bindungsenergie und sind dadurch verhältnismäßig instabil. 


\subsection{Kernmodelle}
\subsubsection*{Tröpfchenmodell}

Hierbei ist die Dichte der Kerne etwa konstant, ebenso die Bindungsenergie pro Nukleon, daraus leitet sich eine Analogie zu einem Flüssigkeitstropfen ab. Weiters wird daraus auf die \textbf{geringe Reichweite der Kernkräfte} geschlossen. Wechselwirkung herrscht nur zwischen benachbarten Nukleonen. Die Bindungsenergie entspricht aber nicht dem Potentialminimum und ist unabhängig von der Kernmasse und beträgt im Mittel ca. \textbf{8MeV}.  

\subsubsection*{Einzelnukleonen-Modell (Schalenmodell)}
Es gibt \textbf{Wechselwirkungen} zwischen Nukleonen, die Eigenschaften sind
\begin{itemize}
\item Kerndrehimpuls $\vec{L_I}$
\item magnetisches Kernmoment $\vec{\mu_I}$
\item elektrische Quadrupolmomente Q
\item empirische Auswahlregeln des $\beta$-Zerfalls
\item "magische Zahlen" 
\end{itemize}
Magische Protonen- und Neutronenzahlen wie 2,8,14,20,...,50,82,126 haben große relative Häufigkeit, große Stabilität und viele Isotope. Kerne mit
\begin{itemize}
\item N = (magische Zahl - 1 ) besitzen großen Wirkungsquerschnitt für Neutroneneinfang
\item N = magische Zahl haben besonders kleinen Wirkungsquerschnitt
\item N = (magische Zahl + 1) darunter die einzigen bekannten Neutronenstrahler
\end{itemize}
Man stellt sich den Kern aus Schalen aufgebaut vor analog zur Elektronenhülle, wobei Quantenzahlen die Nukleoneneigenschaften kennzeichnen, pro Schale $2 \cdot n^2$ Plätze für Nukleonen und es gilt das PAULI-Prinzip, Kerne, die gerade Protonen- und Neutronenzahlen besitzen (gg-Kerne) sind wegen der gegenseitigen Spinabsättigung besonders stabil, weniger stabil sind dabei gu- und ug-Kerne. Am instabilsten sind dabei uu-Kerne. 
Die Spin-Bahn-Kopplung ist bei Nukleonen viel ausgeprägter als bei den Hüllenelektronen, es kommt daher zu einer Energieaufspaltung. \textbf{Magische Zahlen} liegen vor, wenn der \textbf{Energieabstand} zum nächstmöglichen Zustand besonders \textbf{groß} ist. \\
Zusammenfassend ist also eine \textbf{Erklärung der wesentlichsten Kerneigenschaften} durch \textbf{zwei völlig verschiedene, miteinander inkompatiblen Kernmodellen} möglich. 

\subsection{Natürliche Radioaktivität}
Es gibt \textbf{4 natürliche Zerfallsreihen}: Thorium, Neptunium, Uran-Radium und Uran-Actinium, es gibt dabei \textbf{3 verschiedene Zerfallsarten}
\begin{itemize}
\item \textbf{$\alpha$-Zerfall}: Vom Magnetfeld wenig abgelenkt, positive Helium Kerne, es ändert sich \\
MZ $\Rightarrow$ MZ - 4 + $\alpha$ und Z $\Rightarrow$ Z-2
\item \textbf{$\beta$-Zerfall}: stärkere magnetische Ablenkung und es besteht aus schnellen Elektronen bzw. $\beta$-Teilchen \\
MZ bleibt gleich, Z $\Rightarrow$ Z+1
\item \textbf{$\gamma$-Zerfall}: keine magnetische Ablenkung, besteht aus sehr kurzwelliger elektromagnetischer Strahlung, Strahlung vergleichbar mit RÖNTGENstrahlung, aber noch mehr Energie, es ändert sich weder die Massen- noch die Kernladungszahl, es handelt sich um Übergänge zwischen verschiedenen Energiezuständen des selben Kerns. 
\end{itemize}

\subsubsection*{Zerfallsreaktion und Zerfallsgesetze}

Beim Zerfall entsteht ein Kern K' mit geringerer Z/MZ + emittiertes Teilchen x + Energie $\Delta E$, diese $\Delta E$ verteilt sich auf K' und x und möglicherweise $\gamma$-Quanten. \\
Der radioaktive Zerfall ist ein \textbf{statistischer Vorgang}, wie bei den optischen Niveaus der Atome kann daher nur eine \textbf{mittlere Lebensdauer eines Kerns} angegeben werden. Die \textbf{Halbwertszeit} ergibt sich zu
\begin{equation}
\tau_H = \frac{ln(2)}{\lambda}
\end{equation}
Nach verstreichen dieser Zeit ist also die Hälfte der vorhanden Kerne zerfallen, die Zerfallskurve folgt also einer e-Potenz. \\
die Anzahl der Zerfälle pro Zeiteinheit bezeichnet man als \textbf{Aktivität} A
\begin{equation}
A(t) = A_0 \cdot e^{-\lambda \cdot t}
\end{equation}
Als Einheit wird \textbf{Bequerel}(Bq) verwendet
\begin{equation}
1 \text{Bequerel (Bq)} = 1\frac{\text{Zerfall}}{\text{Sekunde}}
\end{equation}

\pagebreak

\subsubsection*{$\alpha$-Zerfall}
Von natürlichen radioaktiven Strahlern emittierte $\alpha$-Teilchen besitzen eine Energie zwischen 4 und 9 MeV. In Luft haben diese eine \textbf{Reichweite} von einigen Zentimetern, das beobachtete Energiespektrum ist diskret, der Kern kann aus seinem Grundzustand oder einem angeregten Zustand strahlen. \\
Ein vom Kern emittiertes $\alpha$-Teilchen müsste klassisch eine Energie besitzen, die höher als die Potentialbarriere ist, mit gewisser Wahrscheinlichkeit ist aber auch ein "durchtunneln" möglich. Dort gelangt das Teilchen dann in das abstoßende COULOMB-Feld der Kerns, wo es beschleunigt wird. Bei höherer Energie des $\alpha$-Teilchens existiert eine höhere Tunnelwahrscheinlichkeit, die Zerfallskonstante $\lambda$ ist größer, ebenso die kinetische Energie des Teilchens. \\
Die kinetische Energie infolge der COULOMB-Abstoßung lautet, woraus sich Rückschlüsse auf $r_0$ ziehen lassen
\begin{equation}
E_{kin} = E_{pot} = \frac{2 \cdot (Z-2)\cdot e^2}{4\pi \cdot \epsilon_0 \cdot r_0}
\end{equation}


\subsubsection*{$\beta$-Zerfall}
Obwohl Ausgangs- und Endkern definierte Energie besitzen, zeigt das \textbf{Energiespektrum} von $\beta$-Teilchen einen \textbf{kontinuierlichen Verlauf}, dies steht im Widerspruch zum Energieerhaltungssatz, die Zerfallsenergie sollte einen diskreten Wert besitzen. \\
Daraus schließt man, dass außer dem $\beta$-Teilchen noch ein \textbf{Neutrino} emittiert wird. Das Neutrino ist notwendig zur \textbf{Erfüllung von Erhaltungssätzen}:
\begin{itemize}
\item \textbf{Energieerhaltung}
\item \textbf{Impulserhaltung}: Das Neutrino kann aus dem Rückstoß der zerfallenen Kerne erahnt werden
\item \textbf{Spinerhaltung}: Kerne mit halbzahligem Spin besitzen nach einer $\beta$-Umwandlung wieder halbzahligen Spin
\end{itemize}
Nach der Entdeckung des Neutrons konnte man diese Emission über den Zerfall des Neutrons anders deuten
\begin{equation}
n \Rightarrow p + e^- + \overline{\nu_e}
\end{equation}
wobei $\overline{\nu_e}$ als \textbf{Antineutrino} bezeichnet wird. \\
Die Umkehrreaktion ist (mit $\nu_e$ Neutrino), ist eine endotherme Reaktion, läuft also nicht freiwillig ab
\begin{equation}
p \Rightarrow n + e^+ + \nu_e
\end{equation}

\subsubsection*{$\gamma$-Strahlung}
Die $\gamma$-Strahlung wird gedeutet als \textbf{Emission elektromagnetischer Strahlung} beim Übergang zwischen verschiedenen Energiezuständen eines Kerns. 
Ein anderer Kern kann ein $\gamma$-Quant absorbieren, wenn dieses die innere Energie $E_2-E_1$ an den Kern überträgt, das $\gamma$-Quant muss aber höhere Energie besitzen $h \cdot \nu' = E_2 - E_1 + E_{Rest}$, daher stimmen die Emissions- und Absorptionsfrequenz derselben $\gamma$-Linie nicht überein, es ist also keine Selbstabsorption emittierter $\gamma$-Quanten möglich. 


\subsection{Nachweis- und Messmethoden für energiereiche Teilchen}
\subsubsection*{Örtliche Nachweismethoden}
\textbf{Ionisationskammern}: Ein energiereiches Teilchen erzeugt auf seinem Weg durch Materie Ionen, der durch Ionisierung verursachte Anstieg der Leitfähigkeit führt zu Stromerhöhung bei anliegender Spannung. Es wird also die Ionisierung der Luft durch energiereiche Teilchen gemessen (keine $\gamma$-Strahlen werden detektiert). \\
\textbf{Zählrohr}: Hierbei handelt es sich um eine Art Zylinderkondensator, auf einem feinen Draht liegt Hochspannung an, dadurch herrscht auch eine hohe Feldstärke. Ein durch Radioaktivität erzeugtes Ion kann unter Umständen eine "Ionenlawine" auslösen, dann sind einzelne Teilchen zählbar! Es gilt für a = Zylinderdurchmesser und b = Drahtdurchmesser
\begin{equation}
E(r) = \frac{U}{r \cdot ln(b/a)}
\end{equation}
Man kann durch variieren der Spannung U verschiedene Bereiche unterscheiden, für kleine Spannungen arbeitet das Rohr als Ionisationskammer, danach als Proportionalzählrohr, im Betrieb als GEIGER-Zähler ist bei erzeugten Stromimpulsen die Teilchen-Art unbestimmbar und es hat eine sehr hohe Verstärkung. 
Wenn das ionisierende Teilchen eine Ionenlawine loslöst, muss diese Gasentladung durch einen hohen Arbeitswiderstand oder andere Gase wieder "gelöscht" werden. \\
\textbf{Kristallzähler} funktionieren wie Photohalbleiter, die Leitfähigkeit steigt bei Strahlung. 


\subsubsection*{Sichtbarmachen von Bahnkurven}
\begin{itemize}
\item \textbf{Nebelkammer}: In der Nebelkammer befindet sich übersättigter Dampf, durchläuft diesen ein ionisiertes Teilchen so entsteht eine Nebelspur bei entsprechender Beleuchtung. Die Energie kann durch eine vorherige Ablenkung im Magnetfeld bestimmt werden. 
\item \textbf{Blasenkammer}: Entstehung von Dampfbläschen in überhitzten Flüssigkeiten, der Vorteil ist, dass die Teilchenspuren sehr kurz gegenüber Luft sind, es ist also die Untersuchung von sehr energiereichen Teilchen möglich.
\item \textbf{Funkenkammer}: Hierbei handelt es sich um einen verschachtelten Kondensator nahe der Durchbruchspannung, bei ionisierten Teilchen zünden die Überschläge. 
\item \textbf{Drahtkammer}: In einem Gehäuse sind dich Drähte gespannt mit Spannung zwischen diesen, ein eintretendes Ion hinterlässt eine Spur von Ladungsträgerpaaren, der dadurch entstehende Impuls bzw. dessen Laufzeiten werden mit Computern als Bahnen der Teilchen sichtbar gemacht. 
\item \textbf{Kernspurplatten}: Spezialphotoplatten zeigen nach Entwicklung Bahnspuren von Elementarteilchen
\end{itemize}

\pagebreak

\subsubsection*{Integrierende Methode}
Oft sind zeitlich integrierende Strahlungsmessungen erwünscht (zum Beispiel zur Feststellung von Strahlenbelastungen von Menschen). Solche Geräte werden \textbf{Dosimeter} genannt, unterschiedliche Prinzipien werden hier angewandt. 

\subsubsection*{Bestimmung von Ladung, Ruhemasse und Energie}
Über die Reichweite in Gas kann auf die Energie geschlossen werden. Der Bahnradius kann in einem Magnetfeld bestimmt werden, wenn die spezifische Ladung bekannt ist und die Teilchenenergie ist bestimmbar aus der Kernspur nach Vielfachstreuung in Kernspurplatten. 

\subsubsection*{Wechselwirkung radioaktiver Strahlung mit Materie}
\begin{itemize}
\item \textbf{$\alpha$-Teilchen}: Die Maximalenergie ist 8.6 MeV, beim Durchgang durch Materie erleiden sie ihren hauptsächlichen Energieverlust durch Ionisierung, die Reichweite an der Luft sind einige Zentimeter in Gewebe einige Mikrometer. 
\item \textbf{$\beta$-Teilchen}: Die Maximalenergie ist bis zu 16 MeV, sie haben eine hohe Geschwindigkeit und wechselwirken auf drei Arten mit Materie
\begin{itemize}
\item \textbf{Elastische Streuung an Kernen und Hüllen}: Vorallem langsame $\beta$-Teilchen
\item \textbf{Ionisation}: Bewirkt Energieverluste im Bereich von 10 keV bis 2 MeV
\item \textbf{Strahlung}: Beim Abbremsen der $\beta$-Teilchen entsteht RÖNTGENstrahlung
\end{itemize}
Insgesamt gibt es eine exponentielle Schwächung
\item \textbf{$\gamma$-Strahlen}: Es handelt sich um Photonen sehr hoher Energie, auch hier gibt es ein exponentielles Schwächungsgesetz und Absorptions- und Streumechanismen wie bei RÖNTGENstrahlung. 
\end{itemize}



\subsection{Neutronen}
\subsubsection*{Neutronenquellen}

Entdeckt wurde das Neutron durch Mischung von Radiumchlorid als Lieferant von $\alpha$-Teilchen und Be-Pulver, bei der Kernreaktion werden ein Neutron n und ein $\gamma$-Quant emittiert. 
\begin{itemize}
\item \textbf{Kernreaktion mit Neutronenabgabe}
\item \textbf{Fusionsneutronenquellen}: Beschuss von Li, Be, C... mit künstlich beschleunigten Stoßteilchen
\item \textbf{Kernreaktoren}: Die hauptsächliche Quelle für langsame Neutronen. 
\item \textbf{Photoneutronenquellen}: Bestrahlung von D oder Be mit $\gamma$-Quanten aus natürlich strahlenden Radionukliden, liefert monochromatische Neutronen.
\item \textbf{Zerschlagen von Deuterium}: D wird in einem Zyklotron beschleunigt und auf Auffängerscheibe gebremst, D-Kern dissoziiert unter Aufwand der Bindungsenergie und es entsteht dabei ein scharf gebündelter Neutronenstrahl. 
\end{itemize}

\subsubsection*{Thermische und monochromatische Neutronen}
Neutronen sind \textbf{elektrisch neutral} und müssen daher bei der Annäherung an einen anderen Kern keine COULOMB-Barriere überwinden, sie haben bei geringer Teilchengeschwindigkeit eine große Verweildauer in der Nähe anderer Kerne, können sich an dies anlagern und danach Kernreaktionen auslösen. \\
Von \textbf{thermischen Neutronen} spricht man, wenn ihre Geschwindigkeitsverteilung der Umgebungstemperatur entspricht. 
\subsubsection*{Erzeugung monochromatischer Neutronen (Neutronen gleicher kinetischer Energie}
Ein Strahl von Neutronen mit definierter Geschwindigkeit entspricht einer Materiewelle mit einer bestimmten DE BROGLIE-Wellenlänge, eine derartige Welle kann an einem geeigneten Beugungsgitter Interferenzerscheinungen zeigen. 

\subsubsection*{Neutronenausgelöste Kernreaktionen}
Wenn \textbf{schnelle Neutronen} einen Kern treffen, haben sie etwa die gleiche Wirkung wie schnelle Protonen, sie lösen die Emission von n,p,$\gamma$, usw. aus. \\
\textbf{Langsame Neutronen} können wegen der fehlenden Potentialschwelle leicht in schwere Kerne eindringen, dabei wird die Bindungsenergie frei und es bildet sich ein Zwischenkern, der wieder zerfällt. In Spezialfällen erfolgt der Zerfall des angeregten Kerns in große Bruchstücke - \textbf{Kernspaltung}.  

\subsubsection*{Nachweis von Neutronen}
\subsubsection*{Rückstoßkerne} Neutronen werden beim Durchgang durch Materie an den Atomkernen des Materials gestreut. Die streuenden Kerne erleiden beim Stoß mit schnelle Neutronen einen Rückstoß, der an Nebelspuren erkennbar ist. Besonders groß ist die Energieübertragung auf leichte Kerne (Protonen), was zur Detektion von Neutronen in Ionisationskammern und Szintillationszählern ausgenutzt werden kann: Bei Bremsung von Neutronen in organischem Material (z.B.: Paraffin) werden Protonen freigesetzt, die ionisieren können. 
\subsubsection*{Kernreaktionen}
Ein sehr wirksamer Nachweis erfolgt bei der Bor-Reaktion, das $\alpha$-Teilchen ist über seine ionisierende Wirkung nachweisbar, zur Neutronendetektion verwendet man $BF_3$ als Füllgas. Da die Reaktion bevorzugt mit langsamen Neutronen erfolgt, müssen schnelle Neutronen vor dem Zählrohr mit Paraffin abgebremst werden. 


\subsection{Erzwungene Kernumwandlung und künstliche Radioaktivität}
Im Gegensatz zur natürlichen Radioaktivität erfolgt hier die Umwandlung von Kernen mit künstlich erzeugten bzw. beschleunigten Geschossen, die auf umzuwandelnde Kerne treffen und dort aufgrund das Einschießens bzw. des Anlagerns einen Zerfall herbeiführen. Derartige Geschosse können p,n,$\alpha$, d,t,e,Kerne,... sein, auch Anlagerung von Neutronen an stabile Kerne löst eine Reaktion aus. \\

\begin{center}
\textit{\textbf{Ladung und Masse} (abgesehen vom Massendefekt) müssen in den Reaktionsgleichungen \textbf{erhalten bleiben}}.
\end{center}

Bei leichten Atomen bzw. schweren Atomen stimmen die Massen der Atome vor der Reaktion mit den Massen danach nicht überein, die überschüssige Bindungsenergie wird frei. 
\subsubsection*{$\beta^+$-Zerfall, Positron}
Entstehen bei der künstlichen Kernumwandlung Kerne mit zu großem Protonenüberschuss, wird ein sogenannter $\beta^+$-Zerfall beobachtet, es wird dabei ein Teilchen emittiert, das gleiche Ruhemasse, aber entgegengesetzte Ladung wie ein Elektron besitzt und als \textbf{Positron} bezeichnet wird. Dabei ändert sich die Massenzahl nicht, aber die Kernladungszahl verringert sich um 1. Das Positron kann in Gegenwart von Materie nicht frei existieren, es vereinigt sich mit einem Elektron unter Emission zweier Photonen, dies wird als Paarvernichtung bezeichnet. 
$\beta^+$-aktive Kerne besitzen zu großen Protonenüberschuss, daher erfolgt eine Umwandlung von p zu n
\begin{equation}
p + \approx 1 MeV \Rightarrow n + \beta^+ + \nu_e
\end{equation}

Zur \textbf{Erfüllung von Energie- und Impulssatz} muss wieder ein drittes Teilchen, das Neutrino $\nu_e$ eingeführt werden. 

\subsubsection*{Bahnelektroneneinfang (K-Einfang)}
Eine Alternative den Protonenüberschuss abzubauen ist der Einfang von Hüllenelektronen, dadurch entsteht Röntgenemission, da innere Elektronen aufgefüllt werden müssen und zusätzlich wird wieder ein Neutrino emittiert. 


\subsection{Kernspaltung}
Es erfolgt eine Anlagerung von Neutronen an Kerne, was zu einer Spaltung inklusive direkter Neutronen-Emission und $\beta^-$-Umwandlung der Folgekerne führen kann. Emittierte Neutronen lagern sich selbst wieder an Kerne an, was insgesamt dann zu einer \textbf{Kettenreaktion} führen kann. Damit diese Kettenreaktion funktionieren kann, muss die Uranabmessung rößer als die freie Weglänge der Neutronen sein, die dazu notwendige Mindestmasse bezeichnet man als \textbf{kritische Masse}. \pagebreak \\
\textbf{A-Bombe}: Es gibt die Uranbombe und die Plutoniumbombe, Zusammenbringen zweier unkritischer Massen zu einer überkritischen Masse, das Ziel ist dabei, die Aufrechterhaltung der Kettenreaktion für lange Zeit. \\
\textbf{Kernreaktor}: Langsamer, kontrollierter Spaltprozess mit konstanter, abgegebener Leistung, die Spaltungsenergie bzw. kinetische Energie der Spaltprodukte wird als Wärme zum Betreiben von Dampfturbinen verwendet. 
Energie wird aus den Brennelementen geliefert, welche von Moderatoren umgeben sind, welche schnelle Neutronen einfangen. Die Regelstäbe können Neutronen absorbieren und werden je nach Reaktionsfortschritt mehr oder weniger in den Reaktionsbereich eingeführt. 

\subsection{Thermische Kettenreaktion, Kernfusion}
Eine Fusion führt aufgrund der Bindungsenergien zu einem Energiegewinn, damit aber Kerne verschmelzen können, müssen sie sich entgegen der COULOMB-Abstoßung auf sehr kleine Distanzen nähern. Es muss daher eine Initialenergie aufgebracht werden und es ist sehr viel Energie notwendig, damit die Teilchen genügen Bewegungsenergie haben. Es ist generell unwahrscheinlich, dies zu realisieren, aber in Sternen auf Grund des enormen Volumens möglich. 

\subsubsection*{Kernfusionsreaktor}
Im Gegensatz zu Kernspaltungsreaktoren, für die nur begrenzt Brennmaterial zur Verfügung steht, wären für einen Reaktor, der H zu He verschmilzt, keine derartigen Probleme zu erwarten und es gäbe keine radioaktiven Abfälle. Am leichtesten ist dies mit Deuterium und Tritium Kernen möglich, Zündung erfolgt bei sehr hohen Temperaturen (ca $10^8$ und hohe Teilchendichte). Wegen hoher Temperaturen ist das Betriebsgas plasmaförmig und muss in Magnetfeldern gehalten werden. 


\subsection{Höhenstrahlung}
Aus dem Weltall kommt \textbf{durchdringende Strahlung} auf die Erde, die eine Ionisierung der Luft hervorruft, man unterscheidet zwischen
\begin{itemize}
\item \textbf{Primärteilchen}: Hauptbestandteil sind Protonen und teilweise $\alpha$-Teilchen, Elektronen fehlen aufgrund des abschirmenden Erdmagnetfelds.
\item \textbf{Sekundärteilchen}: In den oberen Atmosphäreschichten kommt es zu einer Aufteilung der primären Teilchen in Sekundärteilchen durch Stoßprozesse, unterhalb von 20 km Höhe findet man diese Teilchen überwiegend als $e^-$ und $e^+$ vor. 
\end{itemize}


\subsection{Beschleunigermaschinen für Teilchen}
Um bei der Untersuchtung von Stoßprozessen mit hochenergetischen Teilchen nicht allein auf die Höhenstrahlung angewiesen zu sein, versucht man künstliche Kerngeschosse hoher Energie herzustellen, indem man geladene Teilchen mit Hilfe elektrischer und magnetischer Felder beschleunigt. \\
Die dafür benötigten hohen Spannungen kann man zum Beispiel mit dem VAN DE GRAPH-Generator erreichen (bis 10 MV), wo ein Band Ladungen auf einer Halbkugel sammelt.
\subsubsection*{Einfachbeschleuniger}
Die einfachste Möglichkeit eines Beschleunigers ist ein \textbf{Kanalstrahlrohr}, bei dem die Spannung einfach ausgenutzt wird, beim Tandem-Beschleuniger wird ein Teilchen durch Spannung beschleunigt, an einer Folie umgeladen und von der selben Spannung weiter beschleunigt. 
\subsubsection*{Mehrfachbeschleuniger}
Wiederholtes Ausnutzen einer relativ kleinen Spannung.
\begin{itemize}
\item \textbf{Linearbeschleuniger}: Seriell aufgebaute Scheiben bzw. röhrenförmige Elektroden, die beschleunigen. Mit zunehmender Strecke müssen die Elektroden länger werden, da die Geschwindigkeit ansteigt. Bei hohen Geschwindigkeiten generell benötigt man aber sehr große Gesamtlänge des Beschleunigers. 
\item \textbf{Zirkularbeschleuniger}: Hier wird ein Teilchen mehrfach auf der selben Bahn beschleunigt, im \textbf{Zyklotron} wird ein Ion zwischen zwei Halbdosen in einem wechselnden elektrischen Feld beschleunigt, durch das axiale Magnetfeld wird das Teilchen auf der Bahn gehalten und mit steigendem Bahnradius steigt auch die Geschwindigkeit. 
\item \textbf{Synchrozyklotron}: Hier wird das relativistische Teilchen trotz Massenzuname durch Frequenzmodulation des beschleunigenden Feldes im Takt gehalten.
\item \textbf{Betatron}: Hier gibt es Beschleunigung im Transformationsprinzip, ein veränderlicher magnetischer Fluss erzeugt ein kreisförmiges, elektrisches Wirbelfeld, das Energie auf kreisförmig umlaufende Elektronen überträgt. Hier sind Energien bis zu 300MeV möglich und auch $\gamma$-Strahlung. 
\item \textbf{Synchrotron}: Hier werden Teilchen in einem Führungsfeld auf konstantem Radius gehalten, Stärke des Feldes steigt mit Massenzunahme. Die Beschleunigung erfolgt zwischen Magnetstrecken mit Hochfrequenz-Resonatoren. 
\end{itemize}


\subsection{Elementarteilchen}
Es gibt 4 Arten von Wechselwirkungen zwischen Materie
\begin{itemize}
\item \textbf{Starke Wechselwirkung}: zwischen Nukleonen...
\item \textbf{Elektromagnetische Wechselwirkung}: zwischen geladenen Kernteilchen
\item \textbf{Schwache Wechselwirkung}: zwischen Leptonen und Baryonen
\item \textbf{Gravitations-Wechselwirkung}: Graviton
\end{itemize}
Es gelten die Erhaltungssätze für Energie, Impuls, Drehimpuls, Ladung, Parität, Leptonenquantenzahl, Baryonenquantenzahl, Isospin und Seltsamkeit. 
\subsubsection*{Austausch-Wechselwirkung}
Es gibt Äquivalenz zwischen der Darstellung eines Teilchens im Teilchenbild und im Wellenbild, ebenso gibt es eine äquivalente Darstellung zur Kraftwirkung eines Feldes und Vermittlung von Kräften durch Austausch von Quanten. Alle Kraftwirkungen sind auf Austausch eines Bindeteilchens rückführbar. 

\subsubsection*{Elementare Elementarteilchen}
\begin{itemize}
\item \textbf{Leptonen}: Elektron, Müon, Tanon, Neutrinos
\item \textbf{Quarks}: 18 Arten, Mesonen, Baryonen
\item \textbf{Wechselwirkungsteilchen}: Graviton, Photon, Gluon, Weakon
\end{itemize}

\subsubsection*{Zusammengesetze Elementarteilchen}
Diese werden aus Quarks zusammengebaut, es gibt dabei \textbf{Mesonen} und \textbf{Baryonen}. 
Mesonen sind Quark-Antiquark-Paare, sie werden beobachtet in der Höhenstrahlung und beim Beschuss von Kernen mit energiereichen Teilchen. \\
Baryonen sind in zwei Untergruppen unterteilt, und zwar Nukleonen (Protonen und Neutronen) und Hyperonen (sind schwerer als Protonen bzw. Neutronen), diese sind entweder aus drei Quarks oder drei Antiquarks aufgebaut. 
Die einzige vollständig stabile Konfiguration von Quarks ist das \textbf{Proton}. 


















%\pagebreak

%\section{Fragenkatalog}
%\subsection{Theoriefragen}
%\pagebreak
%\subsection{Rechenfragen}






















 



   
\end{document}