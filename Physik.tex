\documentclass[12pt,a4paper,ngerman]{article}
\usepackage{amsmath}
\usepackage[utf8]{inputenc}
%Layout
\setlength{\topmargin}{1pt} \setlength{\headheight}{0.3cm}
\setlength{\textwidth}{16.5cm} \setlength{\textheight}{23cm}
\setlength{\oddsidemargin}{0cm} \setlength{\evensidemargin}{0cm}
\setlength{\headsep}{2pt} \setlength{\parindent}{0pt}
\setlength{\marginparwidth}{80pt} \pagestyle{plain}

\title{Physik TE \\
Ausarbeitung und Zusammenfassung}
\author{Martin Winter}
\date{09.09.2014}

\begin{document}
  \maketitle
\begin{abstract}
Die folgende Zusammenfassung soll als Hilfsmittel dienen um die Physik TE Prüfung erfolgreich abzuschließen. 
Es werden dabei das Skriptum sowie die Ausarbeitungen von Bernhard Geiger, Matthias Straka sowie der Fragenkatalog von der PBS verwendet, hiermit möchte ich mich bei den Urhebern dieser Dokumente herzlichst bedanken. 
\end{abstract}
\pagebreak
  
\tableofcontents
  
\pagebreak
  
  
%-------------------------------------------------------------------------------
%
% Beginn der Zusammenfassung
%
%-------------------------------------------------------------------------------

\section{Mechanik}
\subsection{Basiseinheiten}
Dieses Thema kommt gerne als Prüfungsfrage, hier sind die wichtigsten gelistet:
\\
\\
\textbf{Länge}: Ein Meter ist die Länge der Strecke, die Licht im Vakuum während der Dauer $\frac{1}{c} = \frac{1}{299792458} $ s durchläuft, c steht dabei für die \textit{Lichtgeschwindigkeit} und beträgt etwas weniger als $300000000$ m/s.
\vspace{0.5cm}\\
\textbf{Zeit}: Eine Sekunde ist das $9192631770$-fache der Periodendauer der elektromagnetischen Strahlung des Übergangs zwischen den Hyperfeinstruktur-Niveaus des von äußeren Feldern ungestörten Cäsium-Isotops $^{133}Cs$.
\\
Länge und Zeit können mit einer Unsicherheit von $10^{-14}$ bestimmt werden, Masse nur mit einer Unsicherheit von $10^{-9}$. 
\vspace{0.5cm}\\
\textbf{Masse}: $1$ kg ist die Masse eines Platin-Iridium-Zylinders, der als Massennormal in Paris aufbewahrt wird.

\subsection{Kinematik des Massenpunktes}
Diese beschäftigt sich mit der Beschreibung von Bahnkurven und dem zeitlichen Ablauf einer Bewegung. \\
\subsubsection{Bewegung auf geradliniger Bahn}
Geschwindigkeit ist definiert als
\begin{equation}
v = \frac{\text{zurückgelegte Wegstrecke } s}{\text{dabei verstrichene  Zeit }  t} = 
\frac{\Delta s}{\Delta t}, [m/s]
\end{equation}
Die Beschleunigung ist definiert als
\begin{equation}
a = \frac{\text{Änderung der Geschwindigkeit }v}{\text{dabei verstrichene Zeitdauer }t} = \frac{\Delta v}{\Delta t}, [m/s^2]
\end{equation}

Der einfachste Fall einer ungleichförmigen Bewegung ist somit die gleichförmig beschleunigte Bewegung
\begin{equation}
v = a \cdot t , \quad s = \frac{a\cdot t^2}{2} = \frac{v\cdot t}{2} \text{ sowie } v = \sqrt{2\cdot a \cdot s}
\end{equation}
\subsubsection{Bewegung auf krummliniger Bahn}
Definition des Winkels im Bogenmaß
\begin{equation}
\varphi = \frac{\text{Kreisbogen }b}{\text{Kreisradius }r} [m/m], \text{Radiant}
\end{equation}
Definition der Winkelgeschwindigkeit
\begin{equation}
\omega= \frac{\text{Änderung des Winkels }\varphi}{\text{dabei verstrichene Zeitdauer }t} = \frac{\Delta \varphi}{\Delta t}
\end{equation}
Bei einer gleichförmigen Bewegung kann man die Anzahl der Umläufe pro Sekunde als Frequenz definieren
\begin{equation}
\nu = \frac{1}{\tau}, [s^{-1}], \text{Hz(Hertz)}
\end{equation}
Dabei wird die Zeit für einen vollständigen Umlauf als \textit{Periodendauer} $\tau$ bezeichnet, bei einer gleichförmigen Bewegung gilt somit
\begin{equation}
\omega = 2 \pi \nu
\end{equation}

\subsection{Dynamik}
\subsubsection{Kräfte}
Eine Kraft wird definiert als
\begin{equation}
\vec{F} = m \cdot \vec{a}, [kg\frac{m}{s^2}] \text{ Newton}
\end{equation}
Newton erkannte, dass zwischen Körpern \textit{Wechselwirkungen} herrschen. Als \textit{Kraft} bezeichnet man die Änderung des Bewegungszustand eines Körpers. Ein Körper ohne Wechselwirkungen (auf den keine Kräfte wirken oder wenn die Vektorsumme aller Kräfte Null ist) wird \textit{frei} genannt und ändert seinen Bewegungszustand nicht.
Kräfte können zum Beispiel gemessen werden durch \textit{Verformung} einer Federwage, es gilt dabei
\begin{equation}
F_x = D(x-x_0), \text{ wobei D...Federkonstante und } x-x_0 \text{...Auslenkung entspricht}
\end{equation}
Ein \textit{Wiegen} von Körper mittels einer Balkenwage wird als \textit{Massevergleich} und nicht als \textit{Kraftbestimmung} bezeichnet. 



  
\end{document}