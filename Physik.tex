\documentclass[12pt,a4paper,ngerman]{article}
\usepackage{amsmath}
\usepackage{amsthm}
\usepackage[utf8]{inputenc}
%Layout
\setlength{\topmargin}{1pt} \setlength{\headheight}{0.3cm}
\setlength{\textwidth}{16.5cm} \setlength{\textheight}{23cm}
\setlength{\oddsidemargin}{0cm} \setlength{\evensidemargin}{0cm}
\setlength{\headsep}{2pt} \setlength{\parindent}{0pt}
\setlength{\marginparwidth}{80pt} \pagestyle{plain}

\title{Physik TE \\
Ausarbeitung und Zusammenfassung}
\author{Martin Winter}
\date{09.09.2014}

\begin{document}
  \maketitle
\begin{abstract}
Die folgende Zusammenfassung soll als Hilfsmittel dienen um die Physik TE Prüfung erfolgreich abzuschließen. 
Es werden dabei das Skriptum sowie die Ausarbeitungen von Bernhard Geiger, Matthias Straka sowie der Fragenkatalog von der PBS verwendet, hiermit möchte ich mich bei den Urhebern dieser Dokumente herzlichst bedanken. 
\end{abstract}
\pagebreak
  
\tableofcontents
  
\pagebreak
  
\renewcommand{\arraystretch}{1.5}
%-------------------------------------------------------------------------------
%
% Beginn der Zusammenfassung
%
%-------------------------------------------------------------------------------

\section{Mechanik}
\subsection{Basiseinheiten}
Dieses Thema kommt gerne als Prüfungsfrage, hier sind die wichtigsten gelistet:
\\
\\
\textbf{Länge}: Ein Meter ist die Länge der Strecke, die Licht im Vakuum während der Dauer $\frac{1}{c} = \frac{1}{299792458} $ s durchläuft, c steht dabei für die \textit{Lichtgeschwindigkeit} und beträgt etwas weniger als $300000000$ m/s.
\vspace{0.5cm}\\
\textbf{Zeit}: Eine Sekunde ist das $9192631770$-fache der Periodendauer der elektromagnetischen Strahlung des Übergangs zwischen den Hyperfeinstruktur-Niveaus des von äußeren Feldern ungestörten Cäsium-Isotops $^{133}Cs$.
\\
Länge und Zeit können mit einer Unsicherheit von $10^{-14}$ bestimmt werden, Masse nur mit einer Unsicherheit von $10^{-9}$. 
\vspace{0.5cm}\\
\textbf{Masse}: $1$ kg ist die Masse eines Platin-Iridium-Zylinders, der als Massennormal in Paris aufbewahrt wird.

\subsection{Kinematik des Massenpunktes}
Diese beschäftigt sich mit der Beschreibung von Bahnkurven und dem zeitlichen Ablauf einer Bewegung. \\
\subsubsection*{Bewegung auf geradliniger Bahn}
Geschwindigkeit ist definiert als
\begin{equation}
v = \frac{\text{zurückgelegte Wegstrecke } s}{\text{dabei verstrichene  Zeit }  t} = 
\frac{\Delta s}{\Delta t}, [m/s]
\end{equation}
Die Beschleunigung ist definiert als
\begin{equation}
a = \frac{\text{Änderung der Geschwindigkeit }v}{\text{dabei verstrichene Zeitdauer }t} = \frac{\Delta v}{\Delta t}, [m/s^2]
\end{equation}

Der einfachste Fall einer ungleichförmigen Bewegung ist somit die gleichförmig beschleunigte Bewegung
\begin{equation}
v = a \cdot t , \quad s = \frac{a\cdot t^2}{2} = \frac{v\cdot t}{2} \text{ sowie } v = \sqrt{2\cdot a \cdot s}
\end{equation}
\subsubsection*{Bewegung auf krummliniger Bahn}
Definition des Winkels im Bogenmaß
\begin{equation}
\varphi = \frac{\text{Kreisbogen }b}{\text{Kreisradius }r} [m/m], \text{Radiant}
\end{equation}
Definition der Winkelgeschwindigkeit
\begin{equation}
\omega= \frac{\text{Änderung des Winkels }\varphi}{\text{dabei verstrichene Zeitdauer }t} = \frac{\Delta \varphi}{\Delta t}
\end{equation}
Bei einer gleichförmigen Bewegung kann man die Anzahl der Umläufe pro Sekunde als Frequenz definieren
\begin{equation}
\nu = \frac{1}{\tau}, [s^{-1}], \text{Hz(Hertz)}
\end{equation}
Dabei wird die Zeit für einen vollständigen Umlauf als \textit{Periodendauer} $\tau$ bezeichnet, bei einer gleichförmigen Bewegung gilt somit
\begin{equation}
\omega = 2 \pi \nu
\end{equation}

\subsection{Dynamik}
\subsubsection*{Kräfte}
Eine Kraft wird definiert als
\begin{equation}
\vec{F} = m \cdot \vec{a}, [kg\frac{m}{s^2}] \text{ Newton}
\end{equation}
Newton erkannte, dass zwischen Körpern \textit{Wechselwirkungen} herrschen. Als \textit{Kraft} bezeichnet man die Änderung des Bewegungszustand eines Körpers. Ein Körper ohne Wechselwirkungen (auf den keine Kräfte wirken oder wenn die Vektorsumme aller Kräfte Null ist) wird \textit{frei} genannt und ändert seinen Bewegungszustand nicht.
Kräfte können zum Beispiel gemessen werden durch \textit{Verformung} einer Federwage, es gilt dabei
\begin{equation}
F_x = D(x-x_0), \text{ wobei D...Federkonstante und } x-x_0 \text{...Auslenkung entspricht}
\end{equation}
Ein \textit{Wiegen} von Körper mittels einer Balkenwage wird als \textit{Massevergleich} und nicht als \textit{Kraftbestimmung}
bezeichnet.
Die auf einen Körper wirkenden Kräfte sind meist vom Ort abhängig, somit kann jedem Punkt im Raum ein Kraftvektor zugeordnet werden, man spricht dann von einem \textit{Kraftfeld}. \\
Das \textbf{Gravitationsfeld der Erde} ist gegeben als
\begin{equation}
\vec{F} = -G \frac{m \cdot M}{r^2}\vec{e_r} \text{ wobei M...Erdmasse, m...Körpermasse und G...Gravitationskonstante}
\end{equation}
und das \textbf{Kraftfeld einer elektrischen Ladung} ist gegeben als
\begin{equation}
\vec{F} = \frac{1}{4 \pi \epsilon_0}\frac{q \cdot Q}{r^2}\vec{e_r}
\end{equation}
dabei erhält man ein Coulombfeld, dass dem Schwerefeld einer kugelförmigen Masse entspricht. \vspace{0.5cm}\\
\textbf{Homogenes Kraftfeld eines Plattenkondensators} Die Feldlinien sind hierbei parallel zueinander und haben die gleiche Richtung, ein solches Feld wird \textbf{homogen} genannt. 

\subsubsection*{Grundgleichungen der Mechanik}

\textbf{1. NEWTON'sches Axiom}:

\begin{verse}
Jeder Körper verharrt im \textbf{Zustand der Ruhe} oder der gleichförmigen, geradlinigen Bewegung, solange \textbf{keine Kraft} auf ihn einwirkt. 
\end{verse}

\vspace{0.5cm}

\textbf{2. NEWTON'sches Axiom}:

\begin{verse}
Eine \textbf{zeitliche Änderung} der Bewegungsgröße ist der bewegenden Kraft, durch die sie verursacht wird, \textbf{proportional} und verläuft \textbf{in Richtung der Kraft.}
\end{verse}

\vspace{0.5cm}

\textbf{3. NEWTON'sches Axiom}:

\begin{verse}
Bei zwei Körpern, die \textbf{nur miteinander}, aber \textbf{nicht mit anderen Körpern} wechselwirken, ist die Kraft $\vec{F_1}$ auf den einen Körper entgegengesetzt gleich der Kraft $\vec{F_2}$ auf den anderen Körper. \\
\begin{center}
\textbf{Actio = Reactio: $F_1 = F_2$}
\end{center}
\end{verse}

\vspace{0.5cm}

Als Maß für den Bewegungszustand führt man eine Bewegungsgröße ein, den \textbf{Impuls}
\begin{equation}
p = m \cdot v , \quad p = [kgms^{-1}]
\end{equation}
Die Impulsänderung wirkt als Kraft und ist definiert als
\begin{equation}
F = \frac{dp}{dt} \text{und wegen } p = m \cdot v \text{ gilt } F = m \cdot \frac{dv}{dt} + \frac{dm}{dt}\cdot v
\end{equation}
Ist die \textbf{Masse zeitlich konstant}, so erhält man wiederum 
\begin{equation}
F = m \cdot a
\end{equation}
Ein \textbf{Kraftstoß} ist eine zeitabhängige Kraft F und ruft eine \textbf{Impulsänderung} hervor. \\
Da auf ein abgeschlossenes System keine äußeren Kräfte wirken, kann man daraus schließen, dass der Gesamtimpuls des Systems erhalten bleibt, es gilt also
\begin{equation}
p_1 + p_2 = const
\end{equation}
\begin{verse}
In einem abgeschlossenen System bleibt der \textbf{Gesamtimpuls konstant}. 
\end{verse}
\textbf{Träge und schwere Masse}: \textit{Trägheit} ist die Eigenschaft einer Masse, in ihrem Bewegungszustand zu verharren, wenn keine Kraft auf sie wirkt. Die Kraft zur Änderung ist proportional zur Masse. Schwere Masse ist das Gewicht hervorgerufen durch die Gravitationskraft. \\
Schwere und träge Masse ist nicht unterscheidbar. 

\pagebreak

\subsubsection*{Der Energiesatz der Mechanik}
Legt ein Massepunkt in einem Kraftfeld F(r) das Wegelement $\Delta r$ zurück, so nennen wir das Skalarprodukt
\begin{equation}
\Delta W = \vec{F}(r) \cdot \Delta \vec{r}, \quad W = [Nm] = \text{Joule}
\end{equation}
die \textbf{mechanische Arbeit}, die von der Kraft F am Massepunkt entlang des Weges geleistet wird. Die Arbeit ist eine \textbf{skalare Größe}, die sich aus dem Skalarprodukt zweier Vektoren berechnen lässt
\begin{equation}
W = \int_{P_1}^{P_2}{\vec{F}(r) \cdot d\vec{r}}
\end{equation}
Die Arbeit pro Zeiteinheit nennt man die \textbf{Leistung}
\begin{equation}
P = \frac{dW}{dt}, P = [J/s] = \text{W(Watt)}
\end{equation}

\vspace{0.5cm}
\textbf{Wegunabhängige Arbeit und konservative Kraftfelder}\\
\begin{verse}
In konservativen Kraftfeldern ist die Arbeit bei der Bewegung eines Massenpunktes auf einem geschlossenen Weg Null und hängt sonst nur von Anfangs- und Endpunkt, nicht aber dem gewählten Weg ab. 
\end{verse}

\begin{equation}
W = \int_{P_1}{P_2}{\vec{F}(r) \cdot d\vec{r}} = E_{pot}(P_1) - E_{pot}(P_2)
\end{equation}
Die Kraft, die notwendig ist, um einen Massenpunkt von $P_1$ nach $P_2$ zu verschieben entspricht also der \textbf{Differenz der potentiellen Energie}. Die potentielle Energie in Abhängigkeit von den Ortskoordinaten wird als \textbf{Potential} bezeichnet.
\begin{equation}
E_{pot}(\vec{r})= m \cdot V_{pot}(\vec{r}) = m \cdot g \cdot h
\end{equation}
Der Nullpunkt der potentiellen Energie ist jeweils Definitionssache, dieser wird also meistens am Boden oder im Unendlichen angenommen. \\
Die Gesamtenergie bleibt im Fall konservativer Kräfte erhalten, es gilt 
\begin{equation}
E = E_{pot} + E_{kin} = const
\end{equation}
Der \textbf{Zusammenhang zwischen Kraft und Potential} bei konservativen Kraftfeldern gibt an, dass eine ortsabhängige Kraft vorhanden sein muss
\begin{equation}
\vec{F}  = -m \cdot grad \ V_{pot}
\end{equation}


\pagebreak


\subsubsection*{Drehimpuls und Drehmoment}
Der Drehimpuls ist definiert als
\begin{equation}
\vec{L} = (\vec{r} \times \vec{p})
\end{equation}
Bei einer Bewegung in einer Ebene zeigt der Drehimpuls immer in die Normalrichtung senkrecht zur Ebene. \\

Das Drehmoment ist definiert als
\begin{equation}
\vec{D} = (\vec{r} \times \vec{F}) = \frac{d\vec{L}}{dt}
\end{equation}
es gilt also auch, dass die \textbf{zeitliche Änderung des Drehimpulses gleich dem wirkenden Drehmoment} ist. Es gelten auch Erhaltungssätze für den Drehimpuls. \\
Der \textbf{Eigendrehimpuls} bezieht sich auf die Rotation um eine Achse durch den Körperschwerpunkt, er ist gegeben durch das Produkt von \textbf{Trägheitsmoment I} und \textbf{Winkelgeschwindigkeit $\omega$}. 
\begin{equation}
\vec{L}_{eigen} = I \cdot \omega
\end{equation}
Der \textbf{Bahndrehimpuls} bezieht sich auf den Nullpunkt des Koordinatensystems und ist identisch mit dem Drehimpuls. 

\subsubsection*{KEPLER'sche Gesetze}

\begin{enumerate}
\item Die Bahnen der Planeten sind Ellipsen, in deren Brennpunkt die Sonne steht \\
\item Der Fahrstrahl (Radiusvektor) überstreicht in gleichen Zeiten gleiche Flächen. \\
\item Die Quadrate der Umlaufzeiten zweier Planeten verhalten sich wie die Kuben der großen Ellipsenhalbachsen. 
\end{enumerate}

\subsubsection*{Stoßvorgänge}
Als \textbf{Stoß} bezeichnet man eine kurzzeitige Kraftwirkung zwischen zwei relativ zueinander bewegten Körpern. \\
\textbf{Elastische Stöße}: Hierbei bleibt die gesamte kinetische energie der stoßenden Körper erhalten. 
Es gilt also neben dem Impulssatz auch der Energiesatz der Mechanik 
\begin{equation}
\frac{m_1\cdot v_1^2}{2} + \frac{m_2 \cdot v_2^2}{2} = \frac{m_1 \cdot v_1^{\prime2}}{2} + \frac{m_2 \cdot v_2^{\prime2}}{2}
\end{equation}
\textbf{Unelastische Stöße}: Hierbei gilt zwar der Impulssatz, aber nicht der Energiesatz der Mechanik, ein bestimmter Energieanteil wird also in Wärme oder eine andere Form umgewandelt. Es kommt also ein Summand Q dazu, der Verluste widerspiegelt. 

\begin{equation}
\frac{m_1\cdot v_1^2}{2} + \frac{m_2 \cdot v_2^2}{2} = \frac{(m_1 + m_2)  v^{\prime2}}{2} + Q
\end{equation}

\pagebreak

\section{Elektrizitätslehre}
\subsection{Elektrostatik}
Alle Ladungen sind Vielfache der \textit{Elementarladung}, die Einheit ist Coulomb: 1 C = 1 As. Die Kraft zwischen zwei Ladungen ist definiert als
\begin{equation}
\vec{F} = \frac{1}{4 \pi \epsilon_0}\frac{Q_1 \cdot Q_2}{r^2} \vec{r}_e, \quad \epsilon_0 = 8.854\cdot 10^{-12}\frac{As}{Vm}
\end{equation}
$\epsilon_0$ wird dabei als \textbf{Dielektrizitätskonstante} bezeichnet. \\
Die Kraft auf eine Einheitsladung wird bezeichnet als \textbf{elektrische Feldstärke}
\begin{equation}
\vec{E} = \frac{1}{4 \pi \epsilon_0} \frac{Q}{r^2}\vec{r}_e \left[\frac{V}{m}\right]
\end{equation}
\begin{center}
\textit{
Eine Punktladung Q erzeugt den Fluss $\Phi_{el} = \int_A{\vec{E}\cdot d\vec{A}} =  \frac{Q}{\epsilon_0}$ durch eine die Ladung umgebende Kugeloberfläche. Ist in einer geschlossenen Fläche keine Ladung enthalten, so ist auch der Fluss Null. } 
\end{center}
Die im Raum verteilten Ladungen sind Quellen und Senken des elektrischen Feldes. Da es sich beim Kraftfeld um ein konservatives Kraftfeld handelt, kann man eine Funktion aufstellen, die man als \textbf{elektrostatisches Potential} bezeichnet. 
\begin{equation}
V(P) = \int_{P_1}^{\infty}{\vec{E}\cdot d\vec{s}}
\end{equation}
Die Potentialdifferenz zwischen zwei Punkten im elektrischen Feld nennt man die Spannung U
\begin{equation}
U = V(P_1) - V(P_2) = \int_{P_1}^{P_2}{\vec{E}\cdot d\vec{s}}, \quad U = \text{V(Volt)}
\end{equation}
Die Feldstärke lässt sich auch durch \textbf{Gradientenbildung} aus dem Potential berechnen
\begin{equation}
\vec{E} = -grad \ V
\end{equation}

\subsubsection*{Kräfte auf einen Dipol}
In einem \textbf{homogenen Feld}, in dem sich ein Dipol mit zwei unterschiedlichen Ladungen befindet, sind die Kräfte der beiden Enden entgegengesetzt gerichtet $\Rightarrow$ Drehmoment, dass das Dipol parallel nach dem Feld ausrichtet. 

\subsubsection*{Leiter im elektrischen Feld}
Zwei entgegengesetzte Leiterplatten werden als \textbf{Kondensator} bezeichnet, es gilt dabei 
\begin{equation}
Q = C \cdot U, \quad C = \epsilon_0\frac{A}{d} \ [F]\text{(Farad)}
\end{equation}
Für die Energie im elektrischen Feld erhält man die Gleichung
\begin{equation}
W = \int{V \cdot dQ} = \frac{1}{2}C \cdot U^2
\end{equation}

\pagebreak

\subsubsection*{Millikan Versuch zur Ladungsbestimmung}
Es wurden dabei Öltröpfchen zwischen zwei horizontal angebrachte Kondensatorplatten gebracht, einige Tröpfchen wurden durch Reibung elektrisch geladen und trugen dadurch die gequantelte Ladung $q = n \cdot e$. Im feldfreien Kondensator sinken die Tröpfchen, erst durch Anlegen einer Spannung können diese in der Schwebe gehalten werden und geben dadurch Aufschluss auf deren Ladung, da sich Schwerkraft und elektrische Kraft aufheben müssen. 
Nach Messung der Größe der Tröpfchen lässt sich die \textbf{Elementarladung e} berechnen ($e = 1.602\cdot 10^-19 C$).

\subsubsection*{Ablenkung von $e^-$ in elektrischen Feldern}
Ein Teilchen mit Ladung \textit{q} fliegt mit der Geschwindigkeit $v$ durch zwei horizontale Kondensatorplatten mit der Länge L und wird dabei abgelenkt. 
\begin{equation}
\Delta z = \frac{a \cdot t^2}{2} = \frac{q \cdot E}{m}\frac{L^2}{2v^2} = \frac{E\cdot L^2}{2v^2} \cdot \frac{q}{m}
\end{equation}
Moleküle besitzen elektrische Dipolmomente aufgrund der Verschiebung von Ladungsschwerpunkten. Sie richten sich daher in einem elektrischen Feld aus. Atome selbst besitzen kein Dipolmoment, es wird aber eines im elektrischen Feld induziert. \\
Im elektrischen Feld gibt es keine geschlossenen Feldlinien, es gilt also
\begin{equation}
\oint{\vec{E}\cdot d\vec{s}} \Rightarrow rot \ \vec{E} = 0
\end{equation}
\begin{center}
\textit{Das \textbf{elektrische Feld} ist \textbf{wirbelfrei} und somit ist $\vec{E}$ ein \textbf{konservatives Potential V} zuordenbar.}
\end{center}

\subsection{Stationäre elektrische Ströme}
Den Transport von elektrischen Ladungen nennt man Strom
\begin{equation}
I = \frac{dQ}{dt} = \int{\vec{j} \cdot d\vec{A}}
\end{equation}
wobei $j$ die \textbf{elektrische Stromdichte} ist. Die Kontinuitätsgleichung besagt, dass elektrische Ladungen weder erzeugt noch zerstört werden können. \\
Als Ladungsträger kommen hauptsächlich Elektronen und positive sowie negative Ionen in Frage. Bei elektronischen Leitern sind es hauptsächlich Elektronen, bei Ionenleitern hauptsächlich Ionen und bei gemischten Leitern kommen beide vor. Die \textbf{technische Stromrichtung} ist definiert als die Flussrichtung der \textbf{positiven Ladungsträger}, der technische Fluß geht also immer von \textbf{Plus nach Minus}.
Ohne äußeres Feld bewegen sich Ladungsträger ungeordnet und haben eine mittlere Geschwindigkeit von 0. Beim Ladungstransport stoßen Ladungsträger zusammen und geben dabei Energie ab, entlang eines Leiters findet stets ein Spannungsabfall statt, der die nötige Feldstärke zum Transport zur Verfügung stellt. \\
Für die Leistung gilt
\begin{equation}
P = U \cdot I = \frac{U^2}{R} = I^2 \cdot R
\end{equation}

\subsubsection*{Stromquellen}
Diese basieren auf einer Trennung von positiven und negativen Ladungen, bei dieser Trennung muss Arbeit geleistet werden. Räumlich getrennte Ladungen haben eine Potentialdifferenz, verbindet man diese beiden so fließt ein Strom, der vom äußeren Widerstand bestimmt wird.

\subsection{Statische Magnetfelder}
Magnete haben \textbf{immer Dipolcharakter}. Im \textbf{homogenen Magnetfeld} haben magnetische Dipole keine resultierenden Kräfte. Im \textbf{inhomogenen Feld} wirkt ein \textbf{Drehmoment} nd auch eine Kraft $\vec{F} = -\vec{p}_m\cdot \vec{B}$. \\
Für die Wirkung zwischen zwei Permanentmagneten gilt
\begin{equation}
\vec{F} = \frac{1}{4 \pi \mu_0}\frac{p_1 \cdot p_2}{r^2}\cdot \vec{r}_e \text{ mit } \mu_0 = 4\pi \cdot 10^{-7} \frac{Vs}{Am}
\end{equation}
Dabei ist $p$ die sogenannte Polstärke und $\mu_0$ die \textbf{magnetische Permeabilitätskonstante}. \\
Alle \textbf{magnetischen Feldlinien} sind \textbf{geschlossen}, es gilt also $div \ \vec{B} = 0$ und der Fluss durch eine geschlossene Fläche ist Null. 
\begin{center}
\textit{Das \textbf{statische elektrische Feld} ist also \textbf{wirbelfrei}, während das \textbf{statische magnetische Feld} Wirbel besitzt.}
\end{center}
Um einen stromdurchflossenen Leiter bildet sich ein Magnetfeld (mit der Rechtsschraubregel). Ein zu einer Spule gewickelter Leiter erzeugt ein annähernd homogenes Magnetfeld im Inneren. Der magnetische Fluss ist gegeben als
\begin{equation}
\Phi_m = \int_{A}{\vec{B}\cdot d\vec{A}}
\end{equation}
Weiter gilt, dass die magnetische Erregung entlang eines Weges die Summe aller umschlossenen Ströme ist
\begin{equation}
\oint{\vec{H} \cdot d\vec{s}} = I
\end{equation}

\subsubsection*{Kräfte auf bewegte Ladungen im Magnetfeld}

Man beobachtet, dass die Kraft stets senkrecht zum Geschwindigkeitsvektor der bewegten Ladung und senkrecht zum Vektor des magnetischen Feldes wirkt, die Kraft nennt man die \textbf{Lorentz-Kraft}
\begin{equation}
\vec{F} = q\cdot (\vec{v} \times \vec{B})
\end{equation}
Werden Elektronen in einem Leiter geführt, so können sie sich nicht ganz frei bewegen, daher wird auf den Leiter selbst eine Kraft ausgeübt
\begin{equation}
\vec{F} = \int_{L_1}^{L_2}{(\vec{j} \times \vec{B}) \ dV}
\end{equation}

\pagebreak


\subsubsection*{HALL-Effekt}
Die auf die Ladungsträger wirkende Lorentz-Kraft verursacht eine Ablenkung der Ladungsträger senkrecht zum Leiter, wobei sich zwischen den Berandungen des Leiters die Hallspannung $U_H$ aufbaut. Die Kraft auf die Ladungsträger im zugehörigen elektrischen Feld $E_H$ kompensiert die Lorentz-Kraft. Dieser Effekt wird oft zur \textbf{Magnetfeldmessung} eingesetzt (HALL-Sonde). 

\subsubsection*{Zusammenhang zwischen elektrischem und magnetischem Feld}
Das Magnetfeld eines Stromes und die LORENTZ-Kraft lassens ich aus dem COULOMB-Gesetz herleiten. Das Magnetfeld ist eine Änderung des elektrischen Feldes bewegter Ladungen, daher auch \textbf{elektromagnetisches Feld} genannt. 
\begin{table}[h!]
  \begin{center}
    \begin{tabular}{| r |  l |}
    \hline
    elektrisches Feld & magnetisches Feld  \\ \hline \hline
    $rot \ \vec{E} = 0$ & 	$rot \ \vec{B} = \mu_0 \cdot \vec{j}$  \\ \hline
    $div \ \vec{E} = \rho / \epsilon_0$ & $div \ \vec{B} = 0$  \\ \hline
    $\vec{E} = -grad \ \Phi$ & $\vec{B} = rot \ \vec{A}$ \\ \hline
    $\vec{j} = \sigma \cdot \vec{E}$ & $\epsilon_0 \cdot \mu_0 = 1 / c^2$ \\
    \hline
    \end{tabular}
  \end{center}
  \caption{Zusammenhang der bisher verarbeiteten wichtigen Vektorgrößen}
\end{table}

\subsubsection*{Materie in magnetischen Feldern}

Es gilt allgemein
\begin{equation}
B_{Materie} = \mu \cdot B_{Vakuum} = \mu \cdot \mu_0 \cdot H_{Vakuum}
\end{equation}
Dabei nennt man $\mu$ die \textbf{relative Permeabilität} und ähnlich wie beim elektrischen Feld beobachtet man eine magnetische Polarisierung der Materie. \\
\textbf{Diamagnetische Stoffe} besitzen kein permamentes magnetisches Dipolmoment, die induzierten Dipole entstehen so, dass sie dem äußeren Feld entgegengesetzt orientiert sind. \\
\textbf{Paramagnetische Stoffe} besitzen permanente magnetische Dipolmomente, die aber statistisch verteilt sind, dass äußere Feld richtet die magnetischen Dipole teilweise aus, wodurch dir Magnetisierung stark zunimmt. 
\\
\textbf{Ferromagnetische Stoffe} haben permanentes magnetisches Verhalten, es entstehen sogenannte WEISS'sche Bezirke, in denen sich magnetische Dipole parallel ausrichten. Nach Wegnahme des Feldes richten sich die Orientierungen der einzelnen Bezirke wieder statistisch aus, die Magnetisierung hängt von der Vorgeschichte ab (\textbf{Hysterese}).

\pagebreak 

\subsection{Zeitlich veränderliche elektrische und magnetische Felder}
\subsubsection*{Induktion}
Wenn sich ein Leiter in einem magnetischen Feld bewegt, wird eine Spannung induziert. \begin{equation}
U_{ind} = -\frac{d\Phi_m}{dt}
\end{equation}
Der Spannungsstoß ist bei langsamer bzw. schneller Durchführung gleich groß.
\begin{center}
\textit{Ein \textbf{zeitlich veränderndes magnetisches Feld} erzeugt ein \textbf{elektrisches Wirbelfeld} mit \textbf{geschlossenen Feldlinien}. Es gilt ganz allgemein $rot \ \vec{E} = -\frac{d\vec{B}}{dt}$}
\end{center}

\begin{table}[h!]
  \begin{center}
    \begin{tabular}{| c |  c | c |}
    \hline
    Art des elektrischen Feldes & elektrostatisch & Wirbelfeld  \\ \hline \hline
    Quelle & 	Ladungen & erzeugt durch $d\vec{B}/dt$ \\ \hline
    Feldlinien & offen $\Rightarrow$ $rot \ \vec{E} = 0$ & geschlossen $rot \ \vec{E} = -d\vec{B}/dt$    \\ \hline
    Potential & konservativ $\Rightarrow \vec{E} = -grad \ \Phi$ & nicht darstellbar als Gradient \\ \hline
    \end{tabular}
  \end{center}
  \caption{Gegenüberstellung}
\end{table}
 Es ist also völlig verschieden vom statischen elektrischen Feld, dessen Feldlinien offen sind. 

\subsubsection*{Selbstinduktion}
Dies bedeutet, dass eine der erzeugenden Spannung entgegengesetzte induzierte Spannung entsteht, dies hemmt das Ansteigen des Stromes und erzeugt beim Abschalten eine Spitzenspannung. \\
Die im magnetischen Feld enthaltene Energie wird berechnet durch
\begin{equation}
W_{magn} = \frac{I^2\cdot L}{2}
\end{equation}

\begin{table}[h!]
  \begin{center}
    \begin{tabular}{| c | c |}
    \hline
    Elektrisches Feld & Magnetisches Feld  \\ \hline \hline
    $W_{el} = \frac{1}{2} C \cdot U^2$ & $W_{magn} = \frac{1}{2} L \cdot I^2$ \\ \hline
    $u_{el} = \frac{1}{2}\epsilon_0 E^2$ & $u_{magn} = \frac{1}{2}\mu_0 \cdot H^2 = \frac{1}{2\mu_0}B^2$ \\ \hline
    \end{tabular}
  \end{center}
  \caption{Gegenüberstellung}
\end{table}

\subsubsection*{Verschiebungsstrom}
Dieser wird definiert als das Verschieben von Ladungen über (nicht verbundene) Kondensatorplatten. Magnetfelder werden als nicht nur von Strömen erzeugt, sondern auch von sich ändernden elektrischen Feldern. 

\pagebreak

\subsection{Elektromagnetische Schwingungen und Wellen}

Eine Schaltung aus Kondensator $C$ und Spule $L$ bildet einen \textbf{elektromagnetischen Schwingkreis}. $C$ sei mit $W_{el} = \frac{C \cdot U^2}{2}$ geladen. Beim Entladen lädt sich die Spule mit $W_{magn} = \frac{L \cdot I^2}{2}$ auf. Ist ein Widerstand $R$ vorhanden, erfolgt die Schwingung gedämpft. \\
Es gibt 3 Arten, wie ein Schwingkreis schwingen kann

\begin{enumerate}
\item Im \textbf{Schwingfall} handelt es sich um eine echte gedämpfte Schwingung, beim \textbf{aperiodischen Grenzfall} schwingt der Kreis einmal und stoppt danach und beim \textbf{Kriechfall} wird langsam eine Amplitude aufgebaut, die sich aber gleich mit der Dämpfung wieder Null nähert. 
\item Bei der \textbf{erzwungenen Schwingung} wird der Schwingkreis von außen mit der Frequenz $\omega$ angeregt. Im Resonanzfall wird die Schwingungsamplitude hierbei sehr groß. \item Eine \textbf{ungedämpfte elektrische Schwingung} kann durch Zuführung der verlorenen Energie über eine Rückkopplung erreicht werden. Die MEISSNER-Schaltung war eine der ersten brauchbaren Schaltungen dieser Art.
\end{enumerate}
Die Informationsübertragung über elektromagnetische Wellen erfolgt durch Modulation des Signals. Bei der \textbf{Stabantenne} handelt es sich um einen offenen Schwingkreis, der elektromagnetische Wellen abstrahlt. Die optimale Länge der Antenne ist $l = \lambda/2$ und die abgestrahlte Leistung entspricht ungefähr $\omega^4$. Bei der Ausbreitung der Wellen wechseln sich ständig elektrische und magnetische Energie ab $\vec{E} \bot \vec{B}$. 

\pagebreak

\section{Optik}
\subsection{Elektromagnetische Wellen}

Elektromagnetische Wellen bestehen aus \textbf{6 Komponenten}, $E_{x,y,z}$ des elektrischen Feldes und $B_{x,y,z}$ des magnetischen Feldes. Die \textbf{Ausbreitungsgeschwindigkeit} einer Welle ist mit $v = \frac{1}{\sqrt{\epsilon_0 \cdot \mu_o}}$ gegeben, im Vakuum beträgt diese $c \approx 3\cdot 10^8 \ m/s$, Licht ist eine \textbf{elektromagnetische Transversalwelle}. \\
Die \textbf{Energiedichte}[$J/m^3$] ist der Energieinhalt des elektromagnetischen Feldes pro Volumseinheit, die gesamte Energiedichte ist gegeben durch

\begin{equation}
u = u_E + u_B =\epsilon_0 E^2 = \frac{1}{\mu_0}B^2 = \sqrt{\frac{\epsilon_0}{\mu_0}}E \cdot B
\end{equation}
und entspricht dem \textbf{Strahlungsdruck p'}. Die \textbf{Photonenenergie} ist gegeben durch
\begin{equation}
E_{Photon} = h \cdot \nu \text{ mit } h = 6.6 \cdot 10^{-34} [Js]
\end{equation}
und der \textbf{Impuls des Photons} ergibt sich zu 
\begin{equation}
p_{Photon} = \frac{E_{Photon}}{c} = \frac{h\nu}{c}
\end{equation}
Aus der Wellenlehre ist bekannt
\begin{equation}
\lambda\nu = c
\end{equation}
\subsubsection*{Abstrahlungsvorgänge}
Ursache für die Aussendung elektromagnetischer Strahlung ist eine ungleichförmig bewegte Ladung z.B.: linear beschleunigte Ladung. Bei der \textit{Dipolstrahlung} werden Elektronen im Inneren einer Stabantenne periodisch verschoben, es kommt dabei zu einer \textbf{Abstrahlung}. 
Bei der \textit{Abstrahlung von Energie durch Atome} handelt es sich um den Übergang zwischen Energieniveaus, die Energiedifferenz wird abgestrahlt
\begin{equation}
E_{Photon} = h\nu = E_2-E_1
\end{equation}
Auch \textit{heiße Körper} strahlen Energie ab, Ursache ist die Temperaturbewegung der Körperteilchen (in allen Frequenzen = \textbf{Strahlungskontinuum}).

\subsection{Ausbreitung von Licht}
\subsubsection*{Dispersion}
Im Vakuum breitet sich Licht als ebene Welle oder als Kugelwelle mit \textbf{Phasengeschwindigkeit c} aus. In Materie kommt zu $\epsilon$ und $\mu$ noch jeweils ein Relativitätsfaktor dazu vlg. $\epsilon = \epsilon_r \cdot \epsilon_0$. Diese Faktoren sind bei der Berechnung der Phasengeschwindigkeit zu berücksichtigen. 
Der \textbf{absolute Brechungsindex} wird angegeben als
\begin{equation}
n = \frac{c}{\nu} = \sqrt{\frac{\epsilon\cdot\mu}{\epsilon_0\cdot\mu_0}} = \sqrt{\epsilon_r \cdot \mu_r} \approx \sqrt{\epsilon_r} \text{ da } \mu_r \text{ sehr klein ist}
\end{equation}
Die Brechzahl n ist frequenzabhängig $n = n(\nu)$, dies nennt man \textbf{Dispersion}.  
\pagebreak
\\
Bei der Brechung findet eine \textbf{Polarisation} der Materie statt, es gibt hierbei Ionen- und Elektronenpolarisation. Ionen sind schwerer und folgen daher schwerer den Änderungen des elektromagnetischen Feldes, Elektronen dagegen sogar bei sehr hohen Frequenzen. Die Brechzahl steigt mit wachsender Frequenz und abnehmender Wellenlänge. An Resonanzstellen zeigt sich aber ein umgekehrtes Bild. \\
Die \textbf{Phasengeschwindigkeit} wird angegeben mit 
\begin{equation}
\nu = \frac{c}{n}
\end{equation}
\subsubsection*{Fortpflanzung des Lichts in Materie}
Die Welle trifft auf Atome, deren Elektronen mitschwingen und damit der Welle Energie entziehen und gleichzeitig strahlen die Elektronen Energie ab wodurch \textbf{Sekundärwellen} entstehen, welcher der Phase der Primärwellen vor- oder nacheilen können. Normalerweise ist die Sekundärwelle schwächer und wird absorbiert, \textbf{Absorption und Dispersion} sind gleichzeitig vorhanden. 
\subsubsection*{Reflexion und Brechung}
\begin{center}
\textit{Jeder Punkt auf einer \textbf{primären Wellenfront} ist Ausgangspunkt einer Kugelwelle (Sekundärwelle). Die Wellenfront zu einem späteren Zeitpunkt ist die Einhüllende der \textbf{sekundären Elementarwellen}.}
\end{center}

Bei der Reflexion gelten folgende Gesetze

\begin{itemize}
\item Einfallender, reflektierter und gebrochener Strahl liegen in der Einfallsebene (definiert durch den einfallenden Strahl und die Flächennormale)
\item Einfallswinkel = Reflexionswinkel $\Rightarrow \theta_i = \theta_r$
\item SNELLIUS'sche Brechungsgesetz: $\frac{sin \ \theta_i}{sin \ \theta_t} = \frac{n_t}{n_i}$
\begin{itemize}
\item Von optisch dichtem ins optisch dünne Material wird \textbf{vom Lot gebrochen}
\item Vom optisch dünnen ins optisch dicke Material wird \textbf{zum Lot gebrochen}
\item \textbf{Totalreflexion} tritt dann auf, wenn der Grenzwinkel überschritten wird \\($\theta_i > arcsin \ \frac{n_t}{n_i}$)
\end{itemize}
\end{itemize}
\textbf{FERMAT'sche Prinzip}: Das Licht durchläuft stets die kürzeste (optische) Strecke zwischen zwei Punkten. Die Zeit für den Weg nimmt einen Maximalwert an. Die optische Wellenlänge wird definiert durch $s = n \cdot d$ mit n als Brechzahl und d als Weglänge. \\
Der Winkel, bei dem keine Reflexion stattfindet, wird \textbf{BREWSTER- Winkel genannt}
\begin{equation}
\theta_{iB} = \text{arctan } \frac{n_t}{n_i}	
\end{equation}

\pagebreak

\subsubsection*{Streuung}
Die Streuung ist die \textbf{Sekundärstrahlung}, welche zu einem kleinen Teil in alle Raumrichtungen mit der selben Frequenz und Phase wie die Hauptwelle gestreut wird (kohärente Streuung). Die Flußdichte S ist frequenzabhängig und proportional zu $\nu^4$. \textbf{Deswegen wird auch blaues Licht stärker gestreut als rotes Licht (Himmelsblau, Morgen- und Abendrot)}.


\subsection{Geometrische Optik}
Eine reale Welle wird beim Durchgang durch eine Spalt gebeugt, besonders stark, wenn die Abmessungen der beugenden Strukturen mit der Wellenlänge vergleichbar sind. Wird dabei ein Objektpunkt in exakt einen Bildpunkt umgewandelt, so spricht man von einer \textbf{stigmatischen Abbildung}. \textbf{Reale optische Systeme} haben \textbf{Abbildungsfehler} aufgrund von \textbf{geometrischer Unvollkommenheiten} und \textbf{Beugungseffekten}. 
\subsubsection*{Linsen}
Sie bestehen im Allgemeinen aus rotationssymmetrischen Glaskörpern mit sphärischen Grenzflächen. Zur Berechnung von Linsen
\begin{equation}
 \text{Abbildungsgleichung: } \frac{1}{g} + \frac{1}{b} = \frac{1}{f} \text{ und die Größe: }M_T = -\frac{b}{g}
 \end{equation} 
 \begin{itemize}
 \item g = f $\Rightarrow$ b = $\infty$
 \item g = $\infty$ $\Rightarrow$ b = f
 \item f = 2f $\Rightarrow$ b = 2f
 \end{itemize}
 
Achsenferne Punkte werden konstruiert, indem man einen Mittelpunktstrahl, einen Strahl parallel zur Achse und einen Strahl durch den Brennpunkt zeichnet. 
 
\subsubsection*{Spiegel}
Ebene Spiegel erzeugen ein \textbf{virtuelles, aufrechtes, seitenverkehrtes Bild}. Gekrümmte Spiegel haben einen Brennpunkt, in dem sich alle parallel einfallenden Strahlen schneiden. 
\begin{equation}
 \text{Abbildungsgleichung: } \frac{1}{g} + \frac{1}{b} = \frac{1}{f} = -\frac{2}{R}
 \end{equation}
\subsubsection*{Prismen}
Man unterscheidet \textbf{Dispersionsprismen} (Zerlegung eines einfallenden Lichtbündels in verschiedene Wellenlängen) und \textbf{Reflexionsprismen} (Ab- bzw. Umlenkung von Strahlenbündeln). 

\pagebreak

\subsection{Optische Instrumente}

\subsubsection*{Das Auge - Fehlsichtigkeit}
Die Brennweite des Auges ist ca. 17mm, das Auflösungsvermögen $\alpha_{min} = 1'$. Beim \textbf{fehlsichtigen} Auge ist der Brennpunkt des entspannten Auges bei \textbf{Kurzsichtigkeit vor} und bei \textbf{Weitsichtigkeit hinter} der Netzhaut. Als \textbf{Astigmatismus} bezeichnet man unterschiedliche Brecheigenschaften des Auges für waagrechte und senkrechte Ebenen (Korrektur mit Zylinderlinse). Die Brechkraft von Linsen wird in Dioptrien angegeben
\begin{equation}
D = \frac{1}{f}
\end{equation}

\subsubsection*{Lupe}
Diese bewirken eine \textbf{Vergrößerung des Sehwinkels}. 
\begin{equation}
V = \frac{d_0}{f} \text{ mit } d_0 \text{ = 25cm, deutliche Sehweite (per Definition)}
\end{equation}

\subsubsection*{Mikroskop}
Bestehen aus zwei Linsen, das Objektiv erzeugt dabei ein vergrößertes, umgekehrtes Bild und das Okular wirkt als Lupe für das Objektivbild, das Bild liegt in der Brennweite des Okulars.
\begin{equation}
V_{ges} = V_O \cdot V_{Ok} = \frac{L = 160mm}{f_O}\cdot \frac{d_0 = 250mm}{f_{Ok}} 
\end{equation}

\subsubsection*{Fernrohr}
Es gibt Refraktoren mit Linsen und Spiegelteleskope. Das astronomische Fernrohr funktioniert wie ein Mikroskop, aber für $\infty$ weit entfernte Gegenstände. Da das Bild verkehrt ist, wird im Feldstecher ein Aufrichtesystem eingesetzt. Beim GALILEI-Fernrohr wird beim Okular eine Zerstreuungslinse eingesetzt. 

\subsubsection*{Kamera}
Ziel ist die Abbildung von im Vergleich zur Brennweite meist sehr weit entfernten Objekten. Anforderungen an das Objektiv sind unter anderem in großes Öffnungsverhältnis, chromatische Korrektur, Bildfeldebnung, keine Verzerrung und Entspiegelung der Oberfläche. 

\subsubsection*{Diaprojektor}
Hohlspiegel - Licht - Kondensatorlinse - Dia - Objektiv - Leinwand

\subsubsection*{Spektrograph}
Beim \textbf{Prismenspektrograph} ist das dispergierende Element ein Glasprisma. Beim \textbf{Gitterspektrograph} wird die Reflexion am Beugungsgitter statt des Prismas verwendet, es werden dabei erst Beugungen erster oder höherer Ordnung verwendet. 

\subsubsection*{Abbildungsfehler}
\begin{itemize}
\item \textbf{Sphärische Abberation}: Parallele Strahlen werden nicht im Brennpunkt gebündelt
\item \textbf{Astigmatismus}: Die Brennpunkte horizontaler und vertikaler Linien fallen nicht zusammen, beispielsweise bei Zylinderlinsen, sphärischen Linsen und nicht auf der optischen Achse liegenden Objektpunkten
\item \textbf{Bildfeldkrümmung}: Bild liegt auf einer gekrümmten Fläche, nicht einer Ebene
\item \textbf{Verzerrung}: Es kommt zu verzerrten Objektdarstellungen, Korrektur durch Mehrlinsensysteme
\item \textbf{Chromatische Abberation}: Die Lichtfarben bündeln sich nicht im Brennpunkt eines Objektivs und führen zu Farbsäumen im Bild, Korrektur mittels Zerstreuungslinse. 
\end{itemize}

\subsection{Polarisation}

Die Polarisierbarkeit des Lichtes ist nur erklärbar, wenn Licht als \textbf{transversale elektromagnetische Welle} aufgefasst werden kann.  So eine transversale Welle in Richtung z kann beschrieben werden durch
\begin{equation}
\vec{E} = \vec{E}_0 \cdot \text{cos}(k\cdot z - \omega \cdot t)
\end{equation}
$E_0$ kann dabei in zwei Komponenten zerlegt werden (x- und y-Achse), wenn eine der beiden Teilwellen eine Phasendifferenz gegenüber der anderen Teilwelle aufweist, wird sich die räumliche Lage von $\vec{E}_0$ ändern. 
\subsubsection*{Linear polarisierte Welle}
Ist die Phasendifferenz zwischen den Teilwellen ein \textbf{ganzzahliges Vielfaches von $\pi$}, so erhält man eine \textbf{linear polarisierte Welle}. 

\subsubsection*{Zirkular polarisierte Welle}
Ist die Phasendifferenz zwischen den Teilwellen ein \textbf{ganzzahliges Vielfaches von $\pi /2$} und sind die \textbf{Amplituden der Teilwellen gleich groß}, so erhält man eine \textbf{zirkular polarisierte Welle}. Rechtszirkular polarisiertes Licht bedeutet, dass die Spitze des Feldstärkenvektors in Ausbreitungsrichtung eine Rechtsschraube beschreibt. 

\subsubsection*{Allgemeiner Fall: Elliptisch polarisierte Welle}
Im Allgemeinen werden die Amplituden der Teilwellen verschieden groß sein und auch die Phasendifferenzen nicht in die vorherigen Kategorien fallen. Dies nennt man eine \textbf{elliptisch polarisierte Welle}. Im natürlichen Licht sind viele linear polarisierten Wellenzüge vorhanden, das Licht erscheint dadurch \textbf{unpolarisiert}. 

\subsubsection{Polarisatoren}
Diese erzeugen aus natürlichem Licht linear polarisiertes Licht.
\begin{itemize}
\item \textbf{Dichroismus}: Selektive Absorption von schwingenden Komponenten (z.B.: Drahtgitterpolarisator für Mikrowellen). Elektrischer Vektor parallel zu den Drähten $\Rightarrow$ wird zum Schwingen angeregt, \textbf{Welle stark absorbiert}. Elektrischer Vektor senkrecht zu Drähten $\Rightarrow$ keine Absorption. 
\item \textbf{Doppelbrechung}: In manchen Kristallen sind die Lichtwege für unterschiedliche Schwingungsrichtungen anders - optisch \textbf{anisotrop}.
\item \textbf{Reflexion}: Parallel zur Einfallsebene polarisiertes Licht wird transmittiert, normal zur Einfallsebene polarisiertes Licht wird reflektiert.
\end{itemize}

\subsubsection{Erzeugung von Phasenverschiebungen}
Schneidet man doppelbrechende Kristalle parallel zur optischen Achse, nehmen ordentliche und außerordentliche Strahlen denselben geometrischen Weg, besitzen aber verschieden Ausbreitungsgeschwindigkeiten $ \Rightarrow$ Phasendifferenz. \\
Mit einem Plättchen, das eine Phasendifferenz von $\Delta \varphi = k\cdot 2\pi + \pi/2$ erzeugt, wird eine Verschiebung der Wellenzüge um $\lambda/4$ erzeugt $\Rightarrow$ \textbf{zirkular polarisiertes Licht}. \\
Mit einem Plättchen, das eine Phasendifferenz von $\Delta \varphi = k \cdot \pi$ erzeugt, kann mit $\lambda/2$ die Polarisationsrichtung der Welle gedreht werden. 

\subsubsection{Optische Aktivität}
Damit bezeichnet man die Drehung der Polarisationsebene einer fortschreitenden Welle bei Durchgang durch ein Medium, optisch aktive Moleküle können als schraubenförmige "Leiter" angesehen werden, es kommt zu einer Drehung der Polarisationsebene im Schraubensinn. 

\subsubsection{Erzwungene optische Effekte}
\begin{itemize}
\item \textbf{Spaltdoppelbrechung}: Bei dieser wird in einem isotrpen Medium durch mechanische Spannung Doppelbrechung induziert und wird dadurch anisotrop, im Allgemeinen entsteht aus linear polarisierten Licht elliptisch polarisiertes Licht. 
\item \textbf{FARADAY-Effekt}: Drehung der Polarisationsebene von Licht in ansonst isotoper Materie, wenn ein Magnetfeld in Richtung der Ausbreitungsrichtung angelegt wird. Allerdings kehrt sich der Drehsinn um, wenn das Licht in die andere Richtung läuft. Wird angewendet als \textbf{FARADAY-Isolator}, welcher in der Lasertechnik verhindert, dass reflektierte Laserstrahlung in den Laser eintritt und als \textbf{FARADAY-Modulator}, zur Modulation der Lichtintensität, beispielsweise bei der Signalübertragung. 
\item \textbf{KERR-Effekt}: Durch ein elektrisches Feld wird ein optisch isotopes Medium doppelbrechend, die optische Achse ist gleich der Richtung des elektrischen Feldes. Gut geeignet sind Flüssigkeiten, deren Moleküle ein permanentes elektrisches Dipolmoment besitzen. Im elektrischen Feld werden Moleküle parallel ausgerichtet. Andwendung: Zum "Schalten" von Licht bis 10 GHz.
\item \textbf{POCKELS-Effekt}: In bestimmten Kristallen wird durch ein elektrisches Feld Doppelbrechung induziert, die linear zur Feldstärke ist. Wie KERR, allerdings mit linearem Zusammenhang und bis 25 GHz. 
\end{itemize}

\subsection{Interferenz}
Allgemein gilt das Superpositionsprinzip
\begin{equation}
\vec{E} = \vec{E}_1 + \vec{E}_2
\end{equation}
Die Intensität ergibt sich zu 
\begin{equation}
I = c \cdot \epsilon_0 \cdot \langle E^2 \rangle
\end{equation}
Bei Überlagerung kommt es zu Interferenzen, konstruktive Interferenz ist gegeben als
\begin{equation}
I_1 = I_2 \ \Rightarrow \ I_{ges} = 4 \cdot I_1
\end{equation}
Bei destruktiver Interferenz gilt $I_{ges} = 0$. 
\subsubsection*{Kohärenzzeit und Kohärenzlänge}
Kohärenz ist die Eigenschaft von Wellen bei Überlagerung Interferenzmuster zu zeigen. Wellenzüge haben nur eine endliche Länge, die mit der Abstrahlungszeit zusammenhängt, es gilt hierbei 
\begin{equation}
l = c \cdot \Delta t
\end{equation}
Die Zeit $\Delta t $, die eine Welle mit \textbf{unveränderter Phase} schwingt nennt man die \textbf{Kohärenzzeit}, die dazugehörige Länge $l_c = c \cdot \Delta t$ \textbf{Kohärenzlänge}. Dies ist messbar mit einem Interferometer, welches Licht aufteilt und auf zwei unterschiedlich langen Wegen leitet und danach interferieren lässt (Michelson-Interferometer). 
Eine große Bandbreite der Frequenzen schränkt die Kohärenzlänge ein, die Kohärenzlänge von weißem Licht ist sehr klein ($\approx 2\mu m$), während monochromatisches Licht eine viel größere Kohärenzlänge ($\approx 10 cm$) und Laserlicht sogar bis zu 100 m. All das wird unter \textbf{zeitlicher Kohärenz} zusammengefasst.
\\ Die \textbf{räumliche Kohärenz} befasst sich mit Phänomenen, die von der räumlichen Ausdehnung der Lichtquelle herrühren, zum Beispiel die kohärente Beleuchtung eines Doppelspaltes. 
\subsubsection*{Interferometer mit Amplitudenaufspaltung}
Es werden Teilwellen mittels eines Strahlenteilers aus einer Primärwelle erzeugt, diese Teilwellen werden wieder vereinigt an einem zweiten oder auch demselben Strahlenteiler, die Phasendifferenz ist dabei vom Weg abhängig. 
\\
Bei \textbf{weißem Licht} handelt es sich um eine Überlagerung von Teilwellen, um Interferenz beobachten zu können, muss die Phasendifferenz für alle Teilwellen in einem Punkt aufgehoben sein, es sind daher nur wenige Interferenzstreifen sichtbar. \\
An \textbf{planparallelen Schichten} wird ein Strahl von der Oberfläche sowie an der Unterseite reflektiert, es treten parallele Strahlen aus, die durch eine Linse zur Interferenz gebracht werden können. Bei einem bestimmen Winkel tritt \textbf{konstruktive Interferenz} auf, dies nennt man auch \textbf{Interferenzen gleicher Neigung}. \\
Bei \textbf{Interferenz gleicher Dicke} wird die Phasendifferenz durch die unterschiedliche Dicke der Schicht bestimmt. 

\subsubsection*{Reflexionsmindernde Schichten}
\textbf{Entspiegelung} durch eine $\lambda/4$-dicke Schicht, wo der 1. und 2. reflektierte Strahlt destruktiv interferieren, dabei muss allerdings die Amplitude der Teilwellen gleich groß sein. \\
Bei \textbf{Mehrfachschichten} kann eine wirksamere Entspiegelung erreicht werden, die Entspiegelung vergrößert dabei die Intensität der transmittierten Welle.

\subsubsection*{Zweistrahlinterferenz}
Die auf das Interferometer einfallende Welle wird in \textbf{zwei Teilwellen} mit \textbf{annähernd gleicher Amplitude} aufgespalten. Beim MICHELSON-Interferometer bilden zwei Spiegel eine virtuelle planparallele Platte, es wird dabei ein Interferenzmuster in Form von Ringen erzeugt. Es wurde damit der Lichtäther falsifiziert und es kann genutzt werden zur Wellenlängenmessung, zur Längenmessung und zur FOURIER-Transformations-Spektroskopie.  \\
Beim SAGNAC-Interferometer wird das Licht über 4 Wege im Kreis geführt, während der gesamte Versuchsaufbau rotiert. Es kann verwendet werden zur Messung von Winkelgeschwindigkeiten und als Lasergyroskop. 

\subsubsection*{Vielstrahlinterferenz}
Die einfallend Welle wird in viele Teilwellen mit abnehmender Amplitude aufgespalten. Beim FABRY-PEROT-Interferometer werden diese Strahlen mit einer Linse gebündelt und erzeugen ein \textbf{Ringsystem konzentrisch um die optische Achse}. Anwendung findet es in hochauflösenden Interferenzspektroskopen, als Laserresonator oder zur Frequenzselektion in der Laserphysik. 


\subsection{Beugung}

Eigentlich besteht kein prinzipieller Unterschied zwischen \textbf{Beugung} und \textbf{Interferenz}, man spricht von Interferenz, wenn sich im allgemeinen relativ wenige Wellen (2 bis einige 100) überlagern und von Beugung, wenn sich sehr viele Wellen (viele zehntausende) überlagern. \\
Unter Beugung versteht man die Abweichungen von der geradlinigen Ausbreitung, wenn die freie Ausbreitung durch Hindernisse gestört wird. \\
\textbf{FRAUENHOFERsche Beugung}: Beugungserscheinungenin unendlich großer Entfernung vom beugenden Objekt (\textbf{Fernfeld}). z.B.: werden bei einer Linse parallele Strahlen gesammelt und in die Brennebene transformiert. \\
\textbf{FRESNELsche Beugung}: Erscheinung in der Nähe des beugenden Objekts (\textbf{Nahfeld}).

\pagebreak

\subsubsection{Beugung am Spalt}

Jeder Punkt des Spalts ist eine Sekundärwelle, die miteinander interferieren. Bei $\theta = 0$ sind alle Sekundärwellen in Phase, es kommt zu konstruktiver Interferenz und man erhält das \textbf{zentrale Maximum}. \\
Für das erste Beugungsminimum gilt $sin \ \theta_1 = \pm \frac{\lambda}{b}$. Der von der Mitte ausgehende Strahl und der Randstrahl interferieren destruktiv. 
\begin{table}[h!]
   \begin{center}
     \begin{tabular}{| c | c |}
     \hline
     $\theta = 0$ & größte Intensität  \\ \hline 
     $sin \ \theta = \frac{\lambda}{2b}$ & halbe Intensität \\ \hline
     $sin \ \theta = \frac{\lambda}{b}$ & 1. Minimum \\ \hline
     $sin \ \theta = \frac{3 \lambda}{2b}$ & 1. Nebenmaximum \\ \hline
     $sin \ \theta = \frac{2 \lambda}{b}$ & 2. Minimum \\ \hline     
     \end{tabular}
   \end{center}
   \caption{Spezialfälle}
 \end{table} 
 \\
Man beobachtet also \textbf{Minima} (Auslöschung) für 
\begin{equation}
sin \ \theta = k \cdot \frac{\lambda}{b} \text{ mit } k = 1,2,3...
\end{equation}
Bei der \textbf{Beugung am Doppelspalt} gilt für die Maxima der Interferenz (mit a = Spaltabstand)
\begin{equation}
sin \ \theta = k \cdot \frac{\lambda}{a} \text{ mit } k = 1,2,3...
\end{equation}
Hierbei ist die Spaltbreite vernachlässigbar, wird sie dennoch berücksichtigt, wird die zuvor definierte Interferenzstruktur mit der Intensitätsverteilung der Beugung am Einzelspalt moduliert. \\
Bei der \textbf{Gitterbeugung} interferieren sehr viele Teilwellen und es bilden sich deutliche Maxima.

\subsubsection{Auflösungsvermögen optischer Systeme}

Jeder Objektpunkt wird durch ein optisches Element endlicher Öffnung abgebildet. Jeder Bildpunkt hat also mindesten den Durchmesser des zugehörigen Beugungsscheibchens. Als noch unterscheidbar gelten zwei Beugungsscheibchen, wenn die den halben Durchmesser auseinander liegen. Die minimale Auflösung des abbildenden Systems ist 
\begin{equation}
\Delta \varphi_{min} = 1.22 \frac{\lambda}{D} \text{ mit D = 2R (Objektiv-Durchmesser)}
\end{equation}

\pagebreak

\section{Quantennatur des Lichtes und der Materie}
\subsection{Quantennatur des Lichtes}






















 



   
\end{document}